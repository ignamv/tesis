\section{Conclusiones}
En este trabajo explicamos la aplicación, el principio de funcionamiento, la
forma de diseño y el resultado de las mediciones de dos dosímetros.
Ambos diseños fueron elegidos porque representan una tendencia no sólo de
la dosimetría sino de toda la microelectrónica:
la integración de circuitos especializados en procesos CMOS.

Dimos un repaso de la física necesaria para entender el principio de
funcionamiento de ambos dosímetros, centrándonos en la operación del transistor
MOS. Para esto cubrimos sus regiones de operación y resumimos los principales
efectos de la radiación.

Tanto en el diseño del irradiador como del dosímetro APS presentamos los
cálculos Monte-Carlo con los que contrastamos las dosis calculadas
analíticamente. En ambos casos detallamos la geometría simplificada que usamos
para la simulación, ya sea para minimizar el tiempo de cálculo (caso del
irradiador) o por falta de información acerca de las dimensiones (caso del
APS).

Para cada circuito partimos de la topología elegida explicando los valores a
elegir, y las medidas de desempeño que buscamos maximizar. Para eso mostramos
los cálculos y simulaciones que guiaron el proceso de diseño.

Finalmente presentamos las mediciones realizadas sobre los dosímetros
fabricados. Extrajimos un subconjunto de las medidas de desempeño anteriores y
las comparamos con los valores esperados en base a la simulación del diseño
final.

En la discusión evaluamos en detalle los aspectos positivos y negativos de todo
el trabajo y aquellos cambios, mejoras o profundizaciones que quedan abiertos
para trabajos futuros.
