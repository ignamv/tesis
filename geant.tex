\subsection{Geant4}
Geant4 es un conjunto de herramientas de código abierto
para simular la interacción entre partículas y materia.
Muchas instituciones colaboran en su desarrollo,
incluyendo CERN, ESA y Fermilab.
Se usa en múltiples áreas de la física, medicina e ingeniería espacial.
\subsubsection{Estructura del programa}
Geant4 es una librería de C++.
Usando interfaces de esta librería,
uno define clases para establecer:
\begin{itemize}
    \item Fuente de partículas: distribución espacial, de orientación, energía
        y tipo de partícula
    \item Geometría y materiales del detector y su entorno
    \item Registro de las interacciones: qué variables o estadísticos
        almacenar o filtrar (por ejemplo excluyendo ciertos tipos de
        partículas).
\end{itemize}
Luego, uno instancia estas clases y las pasa a funciones provistas por Geant4
que coordinan la simulación.

Es posible hacer que estas clases reciban comandos de un archivo de texto.
Esto permite correr variaciones de una simulación sin tener que recompilar el
programa: sólo hacer falta cambiar parámetros en ese archivo.
Así facilita la experimentación rápida.
\subsubsection{Geometría}
Para definir la geometría usé el modelo del irradiador que creé en FreeCAD
durante la fabricación.
Geant4 permite crear el mundo de simulación a partir de un archivo
en formato GDML\cite{chytracek_geometry_2006}
con la clase \code{G4GDMLParser}.
Escribí una herramienta en Python para convertir archivos 
\code{.fcstd} de FreeCAD
a este formato.
Permite definir el material de cada sólido
(usando el nombre del material en la base de datos de Geant4)
y dónde se quieren registrar las interacciones.

Para tener una interfaz más cómoda creé la clase
\code{GDMLDetectorConstruction}. 
Por ejemplo, con esta secuencia de comandos carga un archivo GDML y define
por nombre los volúmenes del modelo que conforman el detector:
\begin{verbatim}
/gdml/load pvc_pb.gdml
/gdml/sensitivevolume detectorLogical
\end{verbatim}

\subsubsection{Fuente de partículas}
Geant4 provee la clase \code{G4GeneralParticleSource}
que permite definir todo tipo de fuentes:
puntuales o extendidas, con distintas formas de haz y espectros de energía.
Esto se configura mediante un archivo de comandos 
leído en tiempo de ejecución.
Por ejemplo, este fragmento crea una fuente 
con forma de disco que dispara partículas en la dirección z:
\begin{verbatim}
/gps/pos/type Plane
/gps/pos/shape Circle
/gps/pos/radius 3 mm
/gps/ang/type beam1d
/gps/direction 0 0 1
\end{verbatim}
\subsubsection{Registro de las interacciones}
Durante la simulación se dan interacciones en todas partes del modelo.
Sólo nos interesa registrar las que ocurren dentro de un volúmen acotado.
En el caso del irradiador, este volúmen representa al humano que queremos
proteger.
No nos interesan todos los datos intermedios de cómo la radiación atravesó 
la masa de plástico, aire y plomo del irradiador. 
En el APS, el volúmen de interés es la zona desierta de una juntura, 
donde queremos estimar la carga generada.

La opción más básica es sumar toda la energía depositada en el volúmen.
Esto brinda la mínima información necesaria para un cálculo de dosis.
Opté por registrar las coordenadas y energía depositada
por cada interacción en el volúmen.
Esto brindó mayor flexibilidad para el análisis posterior en Python.
Por ejemplo, permite analizar la variación de la dosis con la profundidad.
Las interacciones se colectan en memoria usando la clase \code{G4THitsCollection}.
Al final de cada disparo, se escriben a un archivo de texto
usando funciones provistas por \code{G4AnalysisManager}.
