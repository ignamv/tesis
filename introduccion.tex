\section{Introducción}
Luego del descubrimiento de los rayos X,
la radiación se convirtió en una herramienta
que acumuló numerosas aplicaciones médicas e industriales.
Paralelamente,
se empezó a tomar conciencia de 
los peligros de la exposición a la radiación
y la importancia de cuantificar la dosis que recibe una persona.

Hoy en día,
más y más técnicas médicas de diagnóstico y terapia
exponen al paciente a distintas formas de radiación.
El control de la dosis se logra mediante calibración de la maquinaria
y cálculos Monte Carlo para modelar la propagación de la radiación.

Midiendo la dosis administrada a cada parte del paciente
se reduce enormemente este riesgo.
Al mismo tiempo,
abre la puerta a terapias más efectivas:
es posible planificar cada sesión de radiación
en respuesta al resultado de la anterior,
corrigiendo por fallas de alineamiento, calibración y cambios en el
paciente\cite{wu_application_2006}.

Para obtener esa información hacen falta dosímetros 
que se presten al uso médico.
Esto pasa tanto por sus especificaciones técnicas
(sensibilidad, dosis máxima)
como por su costo,
biocompatibilidad y tamaño.
Un grupo muy prometedor de dosímetros usa técnicas 
provenientes de la fabricación de circuitos integrados 
para obtener dosímetros fácilmente
miniaturizables\cite{holmes-siedle_radfet:_1986}.
Actualmente consisten en circuitos que miden el corrimiento de parámetros de un transistor
especialmente sensible a la radiación, debido a su óxido de compuerta muy
grueso.

Dentro de los dosímetros integrables 
(que se pueden incorporar con otras funciones en un circuito integrado),
hay gran interés en aquellos fabricados usando, sin modificación,
procesos comerciales para circuitos integrados\cite{lipovetzky_field_2013}
\cite{wang_temperature_2005}
\cite{garcia-moreno_floating_2012}
\cite{dulinski_cmos_2004}.
Esto quita la posibilidad de optimizar y controlar 
muchos parámetros del proceso.
A cambio de esa restricción, 
permite integrar circuitería adicional
para procesamiento de señales e interfaz con el mundo exterior,
aprovechando las economías de escala de los procesos estándar.

En este trabajo diseñamos, construímos y caracterizamos
dos dosímetros fabricados en un proceso estándar CMOS de
\SI{0.6}{\micro\meter}.
El primero es un Active Pixel Sensor,
de estructura similar a un pixel del sensor de una cámara digital:
tiene un diodo polarizado en inversa, con una zona desierta de portadores.
La radiación incidente en esta zona genera pares electrón-hueco.
El campo eléctrico separa electrones de huecos,
y la baja concentración de portadores aumenta su tiempo de vida permitiendo
recolectarlos.
Su ventaja sobre dosímetros tradicionales es la capacidad de resetearlo
instantáneamente de manera electrónica,
simplemente descargando el capacitor que acumula la carga generada por
radiación.

El segundo es un Floating Gate,
semejante a una celda de una memoria Flash.
Aquí la radiación genera carga en el aislante que rodea la compuerta de un
transistor MOS.
Este dispositivo es ideal para sensar esa carga acumulada:
en condiciones normales, la corriente de compuerta es despreciable 
y no descarga al nodo donde se está acumulando la carga.
Este dosímetro se destaca por la posibilidad de medir radiación sin suministro de tensión.

Ambos dosímetros explotan la respuesta a radiación de estructuras 
normalmente utilizadas para otros fines. 
Luego de una introducción a la teoría de su funcionamiento,
presentamos el proceso y las consideraciones de diseño,
y los resultados de las mediciones de ambos dosímetros 
comparando con los valores calculados.
