\section{Cálculos Monte-Carlo}
\label{montecarlo}
Las interacciones entre radiación y materia son eventos como
\begin{itemize}
    \item Scattering: transferencia de energía y momento, por ejemplo de un
        fotón a un electrón
    \item Creación de pares: conversión de energía a un par partícula/
        antipartícula, por ejemplo cuando un fotón incide en un núcleo y genera
        un electrón y un positrón.
\end{itemize}
Estos procesos no son deterministas.
Su resultado se describe mediante una distribución de probabilidad, como
\begin{itemize}
    \item probabilidad de decaimiento por unidad de tiempo (\emph{actividad}), o
    \item sección diferencial de scattering $\deriv{\sigma}{\Omega}$:
la probabilidad de scattering por unidad de ángulo sólido en 
una dirección dada.
\end{itemize}
En algunos casos esta falta de determinismo es inescapable.
Por ejemplo, en el entorno espacial hay que tener en cuenta la 
pequeña probabilidad de recibir una partícula dañina de muy alta
energía.

En muchos casos de interés (aplicaciones médicas e industriales)
las variaciones se cancelan debido a la ley de los grandes números.
Entonces alcanza con calcular el valor esperado de las magnitudes que se buscan.

Frecuentemente se necesita predecir la dosis que va a recibir un \emph{detector}
(por ejemplo un circuito o persona)
dada una fuente de radiación y un entorno.
Cada partícula generada por la fuente participa en muchas interacciones 
y es capaz de generar múltiples partículas secundarias.
Por lo tanto, el espacio de estados finales 
partículas$\otimes$radiación tiene una gran dimensionalidad.
Esto imposibilita calcular la función de distribución en todo el espacio.
Por lo tanto, no es factible calcular el valor esperado de dosis 
a partir de la función de distribución.

El método Monte-Carlo\cite{roe_probability_1992} consiste en 
generar muestras aleatorias
del estado final en este espacio de probabilidad,
y calcular los estadísticos a partir de ellas.
Para esto se simula la evolución de una partícula,
eligiendo al azar entre las interacciones posibles de acuerdo con su 
probabilidad. 

Para realizar este muestreo se utilizan distintos paquetes de software. 
Los mismos cuentan con bases de datos de materiales 
que contienen información para cada tipo de interacción.
Esto permite que el usuario se limite a modelar la geometría,
usando software de modelado 3D como FreeCAD.

Usé el toolkit Geant4\cite{allison_geant4_2006},
con las partículas y procesos necesarios para radiación $\beta$ y X,
para calcular dosis en distintas situaciones.
