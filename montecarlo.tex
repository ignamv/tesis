\section{Cálculos Monte-Carlo}
\label{montecarlo}
Las interacciones fundamentales entre radiación y materia son 
cuánticas, y por lo tanto no deterministas.
Su resultado se describe mediante una distribución de probabilidad,
como la sección diferencial de scattering $\deriv{\sigma}{\Omega}$:
la probabilidad por unidad de ángulo sólido de scatterear en % TODO scatterear
una dirección dada.

Frecuentemente se busca predecir la dosis que va a recibir un detector 
(por ejemplo un circuito o persona)
dada una fuente de radiación y un entorno.
Una partícula incidente participa en muchas interacciones y es capaz de generar
múltiples partículas secundarias.
Por lo tanto, el espacio de estados finales partículas$\otimes$radiación es complejo
y la función de distribución es difícil de calcular.
No es factible calcular el valor esperado de dosis a partir de la
función de distribución.

El método Monte-Carlo\cite{roe_probability_1992} consiste en generar muestras
del estado final y calcular los estadísticos a partir de ellas.
Para esto se simula la evolución de una partícula,
eligiendo al azar entre las interacciones posibles de acuerdo con su 
probabilidad. 
Usamos el toolkit Geant4\cite{allison_geant4_2006},
con las partículas y procesos necesarios para radiación $\beta$ y X,
para calcular dosis en distintas situaciones.
