\section{Cálculos Monte-Carlo}
\label{montecarlo}
Las interacciones fundamentales entre radiación y materia son 
cuánticas, y por lo tanto no deterministas.
Su resultado se describe mediante una distribución de probabilidad,
como la sección diferencial de scattering $\deriv{\sigma}{\Omega}$:
la probabilidad de scattering por unidad de ángulo sólido en 
una dirección dada.

Frecuentemente se busca predecir la dosis que va a recibir un detector 
(por ejemplo un circuito o persona)
dada una fuente de radiación y un entorno.
Cada partícula generada por la fuente participa en muchas interacciones 
y es capaz de generar múltiples partículas secundarias.
Por lo tanto, el espacio de estados finales 
partículas$\otimes$radiación es complejo
y la función de distribución es difícil de calcular.
No es factible calcular el valor esperado de dosis a partir de la
función de distribución.

El método Monte-Carlo\cite{roe_probability_1992} consiste en generar muestras
del estado final en este espacio de probabilidad,
y calcular los estadísticos a partir de ellas.
Para esto se simula la evolución de una partícula,
eligiendo al azar entre las interacciones posibles de acuerdo con su 
probabilidad. 

Para realizar este muestreo se utilizan distintos paquetes de software. 
Los mismos cuentan con bases de datos de materiales 
que contienen información para cada tipo de interacción.
Esto permite que el usuario se concentre en modelar la geometría,
con software de modelado 3D como FreeCAD.

Usé el toolkit Geant4\cite{allison_geant4_2006},
con las partículas y procesos necesarios para radiación $\beta$ y X,
para calcular dosis en distintas situaciones.
