\section{Conclusions}
This work explains the general characteristics,
operating principle, design process and measurement results
for two dosimeters.
Both designs were chosen because they fit with a trend that is seen 
not only in dosimetry but also in all microelectronics:
the increasing integration of specialized circuits in standard CMOS processes.

We have reviewed the physics required to understand the operating principle for both dosimeters,
centered on the functioning of the MOS transistor. 
We covered its operating regions and the principal effects of radiation.

While covering the design of the irradiator and of the APS dosimeter,
we presented Monte-Carlo simulations,
and contrasted them with analitical calculations.
In both cases, we went into the simplified simulated geometry,
which was either used to lower the computation time (for the irradiator),
or because geometry information was limited (for the APS).

We explained each circuit beginning with the topology that was chosen,
the design variables that needed to be set,
and the performance measures we were aiming to maximize.
For this purpose we presented the calculations and simulations which were used in the design process.

Finally, we have shown the measurements carried out on the fabricated dosimeters.
We extracted a subset of the previously mentioned benchmarks,
and compared them to the values predicted from simulations.

The discussion section evaluated the highlights and the problems with the work.
It also enumerates the changes, improvements or further inquiries left for future works.
