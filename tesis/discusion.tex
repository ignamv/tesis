\section{Discusión}
\subsection{Uso de procesos CMOS estándar}
La dificultad central en el diseño de dosímetros en procesos CMOS estándar
se debe a que el proceso no está diseñado con este fin.
La sensibilidad del dosímetro depende de manera sutil de la geometría y
composición de las distintas estructuras del circuito.
Para circuitos analógicos convencionales, alcanza con que el proceso controle y
reporte al diseñador distintos parámetros de dispositivos: 
capacidades entre capas, corrientes de fuga, tensiones de umbral de MOS, etc. 
Si bien fue posible diseñar los dosímetros en base a estos parámetros,
es deseable optimizar el diseño con información más detallada 
(posíblemente confidencial) acerca del proceso.
\subsection{Biasing para simulaciones Monte-Carlo}
La simulación Monte-Carlo del irradiador empleó un modelo con simetría esférica
para obtener un estimado de dosis worst-case minimizando el tiempo de cálculo.
El problema se debe a que nos interesa la fracción ínfima de partículas que
escapan del irradiador, y la simulación pierde tiempo simulando aquellas que se
detienen adentro.
Una mejora posible para trabajos posteriores sería utilizar técnicas
de biasing en la simulación.
Esta es una técnica usada en cálculos de protección que permite explorar en más
detalle un tipo de resultado reduciendo el tiempo que se dedica a otros
resultados. Esto se logra sesgando la generación de números aleatorios y luego
corrigiendo los resultados para compensar.
\subsection{Estructuras de protección ESD}
La cantidad de mediciones se vio limitada por la fragilidad de los circuitos,
probablemente dañados por descargas electrostáticas (ESD).
Si bien se incorporaron estructuras de protección a las entradas de reset de
los dosímetros APS, los diseños futuros deberían explorar medidas adicionales.
En particular, una forma de proteger el transistor inyector del dosímetro FG
que permita aplicar las tensiones elevadas de inyección.
\subsection{Layout para minimizar corrientes de fuga}
El layout de los dosímetros no prestó particular atención a los distintos
caminos de fuga presentes en el circuito fabricado,
cuando ambos dosímetros son sensibles a fugas pequeñas ya sea antes o durante
la irradiación.
Iteraciones futuras deberían explorar el uso de guardas y posiblemente cambios
en el circuito que permitan controlar las corrientes de fuga.
\subsection{Curvas de descarga de APS}
Las curvas de descarga medidas en los APS de área grande y chica tienen tiempos
de descarga muy distintos.
Tanto esto como la forma particular de las curvas no tienen explicación
inmediata.
Si bien la medición con LED confirma que responden a la luz incidente,
es deseable explicar la forma detallada de estas curvas para entender el
fenómeno y construir un dosímetro robusto.
\subsection{Adición de un Control Gate al FG}
Resulta interesante desarrollar una variante del dosímetro FG que use una capa
de metalización como Control Gate.
Esto daría mayor flexibilidad a la hora de cargar y medir el dosímetro,
permitiendo por ejemplo medir el corrimiento de la tensión umbral en este
nuevo terminal.
\subsection{Variación anómala de la sensibilidad de FG}
No se pudo ajustar la curva de sensibilidad del dosímetro FG de forma
satisfactoria.
Esto apunta a que el modelo usado para el ajuste no captura algunos procesos
necesarios como creación de estados de interfaz o captura de carga en el óxido.
Esta falla del modelo no impide una calibración del sensor a la curva empírica,
pero sugiere cambiar el diseño para minimizar estos procesos y obtener un
dosímetro menos sensible a variables poco controladas del proceso.
% layout con más cuidado, faltaron guardas
