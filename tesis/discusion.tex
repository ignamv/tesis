\section{Discussion}
\subsection{Use of standard CMOS processes}
The main difficulty in designing dosimeters in standard CMOS processes
is that these processes are not designed for such a purpose.
The sensitivity has a subtle dependence on the geometry and doping of the
various circuit structures.
For ordinary analog circuits, it is enough for the foundry to control and inform the designer about
certain device parameters: threshold voltages, saturation currents, capacitances, leakage currents, etc.
Although we succeeded in designing dosimeters based on this limited information,
it would be ideal to optimize the design using more detailed 
(possibly confidential) process information.
\subsection{Variance reduction in Monte-Carlo simulations}
The irradiator Monte-Carlo simulations used a spherically symmetrical model
in order to minimize runtime while providing a reasonable worst-case dose estimate.
The problem with this kind of simulation is that we are only interested in 
the small fraction of particles which escape the irradiator,
while the simulator wastes time tracking those which never escape.
It would be valuable for further works to use simulation variance reduction.
This is a technique used in radiation protection,
which allows for simulation time to be focused on certain outcomes.
This is done by biasing the generation of random numbers so that interesting outcomes
(e.g. particles which escape) are more likely.
In the end, the results are corrected to reflect how unlikely they are in reality.
\subsection{ESD protection structures}
Measurements were limited by the fragility of the samples,
which probably suffered from ESD (electrostatic discharge) damage.
Although protection structures were used in the APS reset inputs,
future designs should look into additional protections.
In particular, there should be a way to protect the FG injector
while allowing it to reach the elevated injection voltages.
\subsection{Leakage-minimizing layout}
The dosimeters were not laid out with leakage current in mind.
Both of them are sensitive to small leakage currents, whether before or during irradiation.
Future iterations should explore using guard rings or possibly changes to the circuit
which keep leakage currents under control.
\subsection{APS discharge curves}
The small and large APS have very different discharge curves.
This discrepancy has no immediate explanation, and neither does the unusual shape of the curves.
Although the LED measurements confirm that the APS are sensitive to incoming light,
it would be desirable to understand the exact shape of these curves.
This would allow us to better understand the discharge phenomena and build a more robust dosimeter.
\subsection{Adding a control gate to the FG dosimeter}
It is possible to reach a deeper understanding of the FG dosimeter by adding an externally accessible gate.
This can be done by routing metal over the FG, creating a capacitor.
It would allow for new ways of characterizing the device,
for example by measuring the shift in Vt as seen from the control gate.
\subsection{Anomalous variation of the FG sensitivity}
It was not possible to achieve a close fit to the measured FG sensitivity.
This suggests that the model we tried to fit is not capturing all relevant processes
such as interface state creation or oxide charge capture.
We can still calibrate the sensor using the empirical curve.
Nevertheless, it would be better to look for designs which minimize
the influence of physical processes which are outside our control.
This would lead to a dosimeter which is less sensitive to fabrication process variations.
% layout con más cuidado, faltaron guardas
