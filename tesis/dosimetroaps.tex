\section{Active Pixel Sensor dosimeter}
\fig{aps}{esquematicos/aps.pdf}{APS dosimeter schematic}
The APS dosimeter has a structure similar to a pixel in a digital camera's image sensor.
It requires a low voltage supply when it's being irradiated,
which can be applied using a small battery.
Moreover, it can be quickly reset.
This enables its use for measurements with high temporal resolution.

This section starts by explaining the theory behind this dosimeter.
It then goes on to analyze how it was designed and optimized.
Finally, it presents measurements from the fabricated sensor,
and the conclusions that follow from these measurements.
%
\subsection{Principle of operation}
%
The APS functions by measuring the charge generated by radiation
striking a p-n junction's depletion region.
The junction's cathode is electrically isolated,
in order to accumulate the generated charge without leakage
(D1 in \figref{fig:aps}).
Before each measurement,
the circuit is reset by charging the cathode to
V$_{\text{DD}}$ with a brief turn-on of the reset transistor M1.
Radiation striking D1's depletion region interacts with the silicon lattice,
depositing energy through processes such as scattering and ionization (section~\ref{sec:radiacion}).
The resulting photons and secondary electrons create electron-hole pairs,
which drift in opposite directions due to the depletion region's built-in electric field.
Electrons accumulate in the cathode, gradually discharging it down to
\SI{0}{\volt}.
After irradiation,
the cathode voltage is measured through two stages of voltage followers M2 and M3.
These make up a circuit that replicates the input voltage at the output,
with a small offset.
Their purpose is to prevent the voltage measurement apparatus from
drawing charge from the cathode,
which would influence the measured value.
\subsection{Previous work}
Turchetta\cite{turchetta_monolithic_2001} describes an X ray APS dosimeter
fabricated using a standard \SI{0.6}{\micro\meter} CMOS process.
He builds a grid of pixels for use in particle tracking.
He exposes the sensor to X rays from \isotope{Fe}{55}
and \SI{15}{\giga\electronvolt} pions.
He writes the sensitivity in Volts per radiation-generated electron.
This number reaches up to \SI{15}{\micro\volt} per electron,
with an RMS noise equivalent to 15 electrons.
He presents an innovative layout which allows for almost all the wafer surface
to be sensitive to radiation.

Matis\cite{matis_charged_2003} also builds an APS array,
using a standard \SI{0.25}{\micro\meter} CMOS process.
He analyzes various signal processing schemes,
adding the output from multiple pixels
in order to collect all of the charge generated by an incoming particle.
He irradiates the sensor with X rays from
\isotope{Fe}{55},
and \SI{55}{\mega\electronvolt} protons.

Conti\cite{conti_use_2013} uses a commercial CMOS image sensor,
in order to study its response to X rays.
Specifically, he analyzes the 2D distribution generated by incoming photons,
along with other statistics.
His radiation source is a radiotherapy machine,
and he simulates the scattering caused by the patient
by using a block of acrylic as a phantom.
%
\subsection{Fabrication process}
Although fabrication of an integrated circuit comes after design,
all aspects of the design are conditioned by the characteristics of the fabrication process.
Specifically,
\begin{itemize}
    \item permissible voltage ranges,
    \item devices available in the Process Development Kit, and
    \item electrical characteristics of said devices
        (threshold voltages, parasitic capacitances, leakage currents, etc).
\end{itemize}

We designed and fabricated both the APS and FG dosimeters using X-FAB foundry's
XC06 process\cite{x-fab_0.6_2008}.
It is a \SI{0.6}{\micro\meter} technology node
(which roughly corresponds to the minimum CMOS gate length),
and is designed to operate with a \SI{5}{\volt} supply.
Its simplest variant provides one polysilicon layer and two metal layers.
It is possible to pay for additional process steps (which require more masks)
to add features such as low-doped regions for high voltage devices,
or etched windows for photosensitive devices.
\subsection{Reset}
The charge storage node is charged through the drain of a P-channel MOSFET,
whose source is connected to \vdd.
(\figref{fig:aps}).
During irradiation, its gate is driven to \vdd to turn it off.
In order to reset the dosimeter,
the gate is grounded.
This biases the PMOS in saturation,
charging the cathode at constant current and
linearly increasing $V_D$ up to $V_t$.
The PMOS then enters the triode region and starts conducting less current,
asymptotically charging the cathode to \vdd.
The use of a PMOS allows the cathode to reach \vdd,
while an NMOS could only reach $V_{dd}-V_{tn}$ before cutting off.

The reset PMOS is a minimum area transistor,
in order to contribute the minimum possible stray capacitance
to the cathode node.
This increases the sensitivity, as will be explained later.
\subsection{Response to incoming particles}
Each incoming particle deposits an average energy which is a function of
the particle's initial kinetic energy
\cite{berger_response_1969} (section~\ref{montecarloaps}).
A fraction $E$ of this energy goes to electron-hole pair production,
creating a charge
\begin{align*}
    Q &= \frac{qE}{E_i}
\end{align*}
with $q$ the electron charge and $E_i$ the pair production energy,
\SI{3.62}{\electronvolt} in silicon.
This charge shows up with a negative sign in the cathode, which lowers its voltage.
The change in voltage
\begin{align*}
    \Delta V &= \frac{\Delta Q}{C}
\end{align*}
is a function of the cathode node's total capacitance $C$.
This capacitance has contributions from
\begin{itemize}
    \item D1's junction,
    \item M1's drain-body junction,
    \item M2's gate, and
    \item metal lines near the node.
\end{itemize}
The total capacitance can be estimated by using the 
fabrication process specifications,
which include per-unit-area and per-unit-perimeter capacitances.
To this end, we used EDA (Electronic Design Automation) tools
which automatically estimate capacitances
based on the layout geometry
(section~\ref{section:diseno_aps}).
\subsection{Monte Carlo simulations}
\label{montecarloaps}
El dosímetro APS detecta energía depositada 
en una región específica de un circuito integrado.
A fines de simularlo, 
simplificamos la geometría del die en tres regiones:
una superficie de SiO$_2$, un sustrato de Si 
y una zona sensible también de Si (\figref{fig:corteaps}).
\fig{corteaps}{figuras/aps/corte.pdf}
{Corte de la geometría usada para simular el APS en Geant4 (no a escala).}
Las dimensiones se extrajeron del diseño del APS 
y de las especificaciones del proceso de fabricación del chip.

Dado que el uso principal de Geant4 es en física de altas energías,
su configuración por defecto no permite simular electrones secundarios por
debajo de \SI{250}{\electronvolt}.
% DETAIL rango a 250eV ~ nm, qué me importa?
Para obtener precisión a escalas de distancia más chicas,
empleamos una lista de procesos de interacción 
compilada para simulaciones en microelectrónica \cite{Raine201497}.
La misma simula con fidelidad electrones hasta \SI{16.7}{\electronvolt},
cuyo rango en Si es del orden de \SI{0.1}{\nano\meter}.

Los resultados se encuentran en la \figref{fig:energia1electron}.
% FIXME falta este gráfico usando
% ../tesis/figuras/aps/deposicion1electron001.pdf y demás
\fig{energia1electron}{figuras/aps/deposicion1electron_todos.pdf}
{Distribución de probabilidad simulada en Geant4 
    para la energía depositada en el sensor (eje X).
Cada gráfico corresponde a una dada energía cinética de la partícula incidente.
La energía promedio depositada está en la \figref{fig:energiadepositadaaps}.}
El aumento de la respuesta al reducir la energía incidente se debe a que 
los electrones menos energéticos 
tienden a frenarse por completo en el detector,
depositando toda su energía.
Los más energéticos, en cambio, depositan una fracción variable de su energía
total.
La energía depositada promedio se encuentra en la
\figref{fig:energiadepositadaaps}.
\fig{energiadepositadaaps}{figuras/poster/aps_respuesta.pdf}
{Respuesta promedio a un electrón incidente,
en función de su energía inicial.
Se ve que los electrones menos energéticos se frenan completamente en el
detector.
Las cruces son el resultado de simulación con Geant4,
mientras que la línea es un cálculo manual 
con los datos tabulados por NIST en ESTAR\cite{berger_estar_????}.}
Se ve que para el rango de energías de interés,
cada partícula incidente produce un cambio de decenas de mV.

La respuesta a fotones es mucho más reducida. 
En los rangos de energía que nos interesan, producen una variación de tensión órdenes de magnitud por debajo de la de un electrón de la misma energía (\figref{fig:aps_respuesta_foton}).
\fig{aps_respuesta_foton}{figuras/aps/respuesta_fotones.pdf}
{Respuesta del APS a un fotón, calculada con datos tabulados en XCOM\cite{suplee_xcom_2009}.}
\subsection{Fuentes de ruido}
El APS se maneja con corrientes y variaciones de carga y tensión muy
pequeñas.
Por eso es crítico conocer los procesos que introducen ruido, 
y su magnitud.
Este ruido se combina con la señal proveniente de la radiación
y determina la resolución del dosímetro: 
la dosis mínima que es posible resolver por encima del
ruido\cite{taylor_introduction_1997}.
\subsubsection{Corriente de fuga de juntura p-n}
Al polarizar una juntura p-n en inversa, fluye una corriente de
pérdida\cite{sze_physics_2007} con un valor de DC dado por
\begin{align*}
    I&=I_s(e^{\frac{qV}{\eta kT}}-1)
\end{align*}
con $V<0$ la tensión aplicada y $I_s$ y $\eta$ parámetros de fabricación de la
juntura.
El valor instantáneo de $I$ fluctúa debido a la naturaleza discreta de los portadores que
atraviesan la barrera de energía de la juntura.
Esta fluctuación se denomina ruido \emph{shot}, una clase de ruido blanco:
su densidad espectral de potencia,
\begin{align*}
    i^2(f) &= 2q|I|,
\end{align*}
no varía con la frecuencia (hasta frecuencias muy altas).
\subsubsection{Fluctuaciones durante reset}
Durante el reset, se carga la juntura p-n hasta $V_{dd}$ a través de M1.
Cerca de la tensión final, M1 entra en modo triodo.
En esta condición de operación, el canal actúa como una resistencia.
Esto significa que produce ruido de Johnson\cite{baker_cmos_2010}.
Al cargar la capacidad de juntura $C$, 
este ruido produce una varianza en la tensión final dada por
\begin{align*}
    \overline{v^2} &= \frac{kT}C.
\end{align*}
Evaluando esta fórmula con la capacitancia del APS
% TODO: es la capacidad de 4x4 o 40x40?
se llega a una tensión de ruido RMS de \SI{1.1}{\milli\volt}.

Esta incertidumbre en la tensión luego del reset puede eliminarse usando 
la técnica Correlated Double Sampling\cite{white_characterization_1974}:
se mide tensión antes y después de la exposición a la radiación.
Al tomar la diferencia se elimina el ruido de reset,
presente en ambas por igual.

\subsection{Diseño del circuito}
\label{section:diseno_aps}
La topología del circuito quedó determinada por la elección de construir un
dosímetro APS con un par de seguidores para su medición.
El paso siguiente en el diseño fue elegir los tamaños de los distintos
componentes para optimizar el desempeño del sensor.

Para una carga dada, la tensión sobre un capacitor es inversamente proporcional
a su capacidad.
En el APS la carga es generada por la radiación,
y la tensión es la señal cuya magnitud queremos maximizar.
Para esto minimizamos las capacidades parásitas del cátodo de D1,
usando transistores de área mínima para M1 y M2.
Restringimos el largo de las conexiones del cátodo,
y las mantuvimos alejadas de otros nodos.
Aplicamos software de Mentor de extracción de capacidades parásitas al layout
resultante, y obtuvimos una capacidad total en el cátodo de \SI{3.4}{\femto\farad}.

El tamaño del primer MOS seguidor, M2, 
tiene que ser el tamaño mínimo para no cargar capacitivamente
al nodo que acumula carga.
Esto deja libre las dimensiones del segundo MOS seguidor, M3.
Mientras más grande es, mayor es su capacidad de gate
y por lo tanto más va a cargar a la etapa anterior.
Por otro lado va a tener mayor capacidad de corriente
para manejar la carga capacitiva del pad de salida.

Estimamos un tamaño inicial para M3 fijando un largo arbitrario
y variando el ancho para minimizar una figura de mérito.
La figura que elegimos representa el delay producido por los dos seguidores,
\begin{align}
    \tau &= \tau_1+\tau_2 = g_{m2}C_{g3} + g_{m3}C_{\textnormal{pad}}
    \label{eq:delay_seguidor_aps}
\end{align}
con $g_m=\pderiv{I_D}{V_G}$ la transconductancia de un MOS 
y $C_g$ su capacidad de gate.
Minimizando la ecuación~\ref{eq:delay_seguidor_aps}
para una polarización pre-fijada,
llegamos a un $W$ inicial de \SI{409}{\micro\meter}.

El paso siguiente es hacer un cálculo más preciso 
que incluya todos los detalles del funcionamiento del circuito.
Para esto simulamos con SPICE el tiempo de respuesta a un electrón incidente
en función del ancho de M3 (\figref{fig:falltime}).
\fig{falltime}{figuras/aps/falltime.pdf}
{Tiempo de respuesta simulado del buffer en función del ancho del MOS del
    segundo seguidor. 
    $W_{\textnormal{ideal}}$ es el $W$ óptimo cálculado a mano de forma
    simplificada.
}
Se ve que hay poca mejora en el tiempo de respuesta 
al aumentar $W$ por encima de nuestro estimado inicial.
Por eso, teniendo en cuenta las limitaciones de área,
elegimos un ancho total de \SI{400}{\micro\meter} repartido entre 8 canales 
(o sea 8 MOS en paralelo, cada uno de \SI{50}{\micro\meter}).

Las dimensiones finales están en la tabla~\ref{fig:areas_aps}.
Incluímos también una variante con un diodo más grande,
de modo que su capacidad opaque las otras capacidades parásitas.
\begin{table}[h]
    \centering
    \caption{Dimensiones del diseño optimizadas para sensibilidad y tiempo de
    respuesta}
    \begin{tabular}{|c|c|c|c|}
        \hline
        Dispositivo&      W (um)&    L (um)&  Canales\\
        \hline
D1&     4&  4&  1\\
M1&     0.8&    0.6&    1\\
M2&     0.8&    0.6&    1\\
M3&     50& 3&  8\\
        \hline
    \end{tabular}
    \label{fig:areas_aps}
\end{table}

Estas dimensiones se utilizaron tanto para la simulación del circuito en SPICE
como para los cálculos Monte-Carlo.
La combinación de estas dos herramientas nos da una sensibilidad esperada de 
\SI{7.1}{\volt\per\gray}.
El diseño físico final está en la \figref{fig:layoutaps},
y el esquemático en la \figref{fig:aps}.
\begin{figure}[p]
    \centering
    \includegraphics[width=\columnwidth]{figuras/gds/aps/todo.png}
    \caption{Layout del dosímetro APS. 
    El transistor de la derecha es el de salida, del segundo seguidor.
    El resto se ve en más detalle en la \figref{fig:layoutapszoom}.}
    \label{fig:layoutaps}
\end{figure}
\begin{figure}[p]
    \centering
    \includegraphics[width=\columnwidth]{figuras/gds/aps/zoom.png}
    \caption{Vista en detalle del layout del APS, excluyendo el transistor de
        salida.
        A la derecha está el diodo, con el cátodo (nwell) en el centro y el
        ánodo (contactos a sustrato) rodeándolo.
        A su izquierda está el primer MOSFET seguidor y abajo el MOSFET de
        reset.
        Los transistores de la izquierda polarizan al primer seguidor.
    El transistor de reset está conectado a un pad (no visible) a través del
    circuito de protección de la \figref{fig:proteccion5v}.}
    \label{fig:layoutapszoom}
\end{figure}
\fig{proteccion5v}{figuras/aps/proteccion.pdf}
{Circuito de protección para la entrada de reset.
Los diodos sólo conducen si la tensión del pad excede \SI{5}{\volt} o
baja de \SI{0}{\volt}.
Cuando llega un pulso de alta tensión
(por ejemplo debido a una descarga electrostática)
los diodos limitan la tensión que llega al circuito.
Esto evita que se polaricen en directa las junturas drain-body y source-body,
previniendo una falla por latchup (sección~\ref{latchup}).
También evita la ruptura de los óxidos de compuerta de MOS.}
\subsection{Medición}
Tomamos dos dies y los bondeamos a placas adaptadoras de TSSOP28,
un tipo de empaquetado de circuitos integrados de montaje superficial
(figuras~\ref{fig:bondeados1}, \ref{fig:bondeados2} y~\ref{fig:pinout}).
\fig{bondeados1}{figuras/aps/bondeados.jpg}
{Dies bondeados a placa adaptadora SMD.
Los zócalos tienen las patas cortocircuitadas para proteger al die de
descargas electrostáticas durante el transporte y almacenamiento.}
\figp{bondeados2}{figuras/aps/die.jpg}
{Detalle del die fabricado con los dosímetros APS y FG 
(arriba en la columna central)
y otros circuitos.}
\figp{pinout}{figuras/aps/pinout1.pdf}
{Layout del die entero con numeración de los pads bondeados}
\subsubsection{Descarga en oscuridad}
Primero medimos la respuesta del sensor sin luz ni radiación
(figuras~\ref{fig:oscuridad4} y~\ref{fig:oscuridad40}).
\fig{oscuridad4}{figuras/aps/oscuridad4.pdf}
{Curva de descarga en oscuridad del APS de 4x\SI{4}{\micro\meter}.
Resulta de resetear el APS y medir su tensión de salida en oscuridad.}
\fig{oscuridad40}{figuras/aps/oscuridad40.pdf}
{Curva de descarga en oscuridad del APS de 40x\SI{40}{\micro\meter}.
Resulta de resetear el APS y medir su tensión de salida en oscuridad.}
Esto nos muestra la descarga del diodo debido a la corriente de fuga en
inversa.

Se ve en ambas figuras la misma curva 
con escalas distintas de tiempo y de tensión.
Esta variación proviene tanto de las áreas distintas de los dos sensores
como de las variaciones aleatorias entre los MOS seguidores (mismatch).
Cada transistor del die tiene pequeñas variaciones debido a imperfecciones en
la litografía y variaciones aleatorias de dopaje.
Estas variaciones no son tan notables en dispositivos grandes debido a que
se cancelan al promediarse sobre mucha área.

Ya que los seguidores usan varios transistores de área mínima,
son particularmente sensibles a variaciones del proceso:
el mismo error absoluto en las dimensiones del canal produce un mayor error relativo.

% Descarga debería ser lineal
% https://books.google.com.ar/books?id=6Rg7AAAAQBAJ&lpg=PA289&ots=yO1HPv_N4E&dq=reverse%20biased%20diode%20%22discharge%20curve%22&hl=es&pg=PA290#v=onepage&q=reverse%20biased%20diode%20%22discharge%20curve%22&f=false

La curva de descarga es la solución a una ecuación diferencial
no-lineal:
\begin{align*}
    \frac{\partial Q}{\partial V}\frac{dV}{dt} &= -I(V)
\end{align*}
La forma de la curva proviene de la variación 
tanto de la corriente de fuga $I(V)$ 
como de la capacidad del diodo $\frac{\partial Q}{\partial V}(V)$.
Ambas dependen de la tensión aplicada,
debido a la variación del ancho de la zona desierta.
Al caer la tensión en inversa,
la zona desierta se vuelve más angosta.
Esto reduce su volúmen y por lo tanto 
la tasa de generación térmica de pares electrón-hueco.
Por otra parte,
su capacidad es inversamente proporcional a este ancho.
Ambos fenómenos reducen la tasa de descarga,
como se ve al final de ambas curvas.

Por otro lado, hay factores de segundo orden 
que no contemplamos en el análisis.
Por ejemplo, el transistor de reset apagado
no es un circuito abierto perfecto, 
sino que tiene una corriente de pérdida muy pequeña.
Esta corriente va a tender a cargar el cátodo de D1,
enlenteciendo la descarga.
%
\subsection{Iluminación con LED}
Medimos las curvas de descarga iluminando los dies con un LED,
variando su corriente para lograr distintas intensidades de iluminación
(figuras~\ref{fig:led4} y~\ref{fig:led40}).
\fig{led4}{figuras/aps/descarga_led_4.pdf}
{Curva de descarga iluminando con un LED el APS de 40x\SI{40}{\micro\meter}.
La corriente del LED aumenta de \SI{.1}{\milli\ampere} a la derecha hasta
\SI{10}{\milli\ampere} a la izquierda,
con 6 curvas por década.}
\fig{led40}{figuras/aps/descarga_led_40.pdf}
{Curva de descarga iluminando con un LED el APS de 4x\SI{4}{\micro\meter}.
La corriente del LED aumenta de \SI{.1}{\milli\ampere} a la derecha hasta
\SI{10}{\milli\ampere} a la izquierda,
con 6 curvas por década.}
Esto permite observar la compresión en tiempo de la curva de descarga 
con el aumento de la radiación incidente.
No fue posible medir el efecto de otros tipos de radiación
(por ejemplo, $\beta$) por limitaciones de tiempo.
\subsubsection{Ruido medido}
Establecimos que una medición con este dosímetro consiste en promediar 10
muestras de tensión tomadas a una frecuencia de \SI{625}{\hertz} 
(período de muestreo \SI{1.6}{\milli\second}).
Al partir de esta definición,
estamos eligiendo no profundizar en
las características intrínsicas de ruido del sensor
(y su dependencia con la frecuencia, por ejemplo).
En cambio, nos enfocamos en el ruido que va a ver el usuario bajo ciertas
condiciones particulares de uso.

Con esta definición de qué es una medición, 
podemos definir de manera precisa el ruido como la desviación estándar
de ese promedio de 10 muestras.
Más precisamente, la tensión de salida a tasa de dosis constante
es la suma de una componente determinista $V_D$ y una aleatoria $\epsilon$
\begin{align*}
    V = V_D(\dot D, t) + \epsilon
\end{align*}. 
Al tomar la diferencia entre dos muestras consecutivas,
queda una pequeña componente sistemática 
(proporcional a $\Delta t\pderiv{V_D}{t}$) 
sumada a la diferencia entre dos variables aleatorias que suponemos
independientes.
Por lo tanto, 
la desviación estándar de la diferencia entre dos muestras consecutivas
es $\sqrt 2$ veces la desviación estándar presente en una muestra.

Para el análisis, medimos 6 disparos del APS
(cada disparo como las figuras~\ref{fig:led4} y~\ref{fig:led40}), 
totalizando 75000 muestras.
Luego diferenciamos estas mediciones y calculamos la desviación estándar.
Los resultados se ven en las figuras~\ref{fig:ruido4} y~\ref{fig:ruido40}.
% TODO: Las curvas medidas serían led4 y led40? Y pregunto:
% 1) Las miles de muestras corresponden a un disparo o a muchos disparos? Y
% cómo comparan los distintos disparos?
% 2) Cuál es el deltaT?

% TODO De dónde sale la equivalencia? ¿Es este sigma representativo de la
% dispersión en una repetición de disparos? Diría que está restringido a la
% freq de muestreo
\fig{ruido4}{figuras/aps/ruido4.pdf}{Ruido a la salida del APS de
    \SI{4x4}{\micro\meter}, calculado tomando diferencias entre muestras
    consecutivas y
escalando para que represente el ruido en un promedio de 10 valores.
La dosis equivalente se calculó con la sensibilidad proveniente
de las simulaciones SPICE y los cálculos de radiación.}
\fig{ruido40}{figuras/aps/ruido40.pdf}{Ruido a la salida del APS de
    \SI{40x40}{\micro\meter}, calculado tomando diferencias entre muestras
    consecutivas y
escalando para que represente el ruido en un promedio de 10 valores.
La dosis equivalente se calculó con la sensibilidad proveniente
de las simulaciones SPICE y los cálculos de radiación.}
También es posible tomar la desviación estándar para cada disparo,
como se ve en las figuras~\figref{fig:ruido_disparo_4} y~\figref{fig:ruido_disparo_40}.
Así se verifica que todos los disparos tienen una estadística similar.
\fig{ruido_disparo_4}{figuras/aps/std_disparo_aps4.pdf}
{Ruido a la salida del APS de \SI{40x40}{\micro\meter},
calculado usando muestras de disparos individuales.
Los resultados se agrupan muy cerca del valor de la \figref{fig:ruido40}.}
\fig{ruido_disparo_40}{figuras/aps/std_disparo_aps40.pdf}
{Ruido a la salida del APS de \SI{4x4}{\micro\meter},
calculado usando muestras de disparos individuales.
Los resultados se agrupan muy cerca del valor de la \figref{fig:ruido4}.}
Podemos convertir estos valores de ruido en dosis usando la sensibilidad
calculada en~\ref{section:diseno_aps}.
Así llegamos a una resolución de \SI{2.0}{\milli\gray} y \SI{2.3}{\milli\gray}
para el APS de \SI{4x4}{\micro\meter} y \SI{40x40}{\micro\meter},
    respectivamente.
Sin embargo, esto está restringido a 
las condiciones de muestreo de las que partimos,
y es posible hacer un análisis más general de 
la magnitud del ruido en función de la frecuencia.
Esto permitiría elegir otra cantidad de promediado o frecuencia de muestreo 
que logre mayor resolución.

Este análisis del ruido tampoco contempla todas los formas posibles
de usar el APS.
Su validez principal es al realizar una medición Correlated Double Sampling,
restando la tensión final a la tensión luego del reset.
Sin embargo,
también existe la posibilidad de medir durante intervalos largos,
sumando la dosis medida en cada ciclo de reset y descarga.
En este caso, se suma el error en cada disparo para llegar a un valor de ruido
proporcional a la raíz de la cantidad de disparos.

A fin de evaluar la repetitividad de la lectura, comparamos la tensión de salida de los APS un tiempo fijo después de su reseteo, para varios disparos.
El resultado (figuras~\ref{fig:comparacion_salida_disparos_aps4} y~\ref{fig:comparacion_salida_disparos_aps40}) es que hay una diferencia entre disparos de \SI{20}{\milli\volt} en el APS de \SI{4x4}{\micro\meter} y de \SI{12}{\milli\volt} en el de \SI{40x40}{\micro\meter}.
\fig{comparacion_salida_disparos_aps4}
{figuras/aps/tension_salida_comparando_disparos_aps4.pdf}
{Tensión de salida para varios disparos, un tiempo fijo luego de resetear el APS de \SI{4x4}{\micro\meter}.}
\fig{comparacion_salida_disparos_aps40}
{figuras/aps/tension_salida_comparando_disparos_aps40.pdf}
{Tensión de salida para varios disparos, un tiempo fijo luego de resetear el APS de \SI{40x40}{\micro\meter}.}
Si bien es una muestra reducida, la repetitividad de la lectura entre disparos es buena.
En particular, es comparable con el ruido de medición caracterizado en las figuras~\ref{fig:ruido_disparo_4} y~\ref{fig:ruido_disparo_40}.
Las lecturas están acotadas en un intervalo de \SI{20}{\milli\volt} para el dispositivo de \SI{4x4}{\micro\meter} y es aproximadamente la mitad en el dispositivo de \SI{40x40}{\micro\meter}; valores comparables con las desviaciones estándar obtenidas para la dispersión de las lecturas intra-disparos.
%
\subsection{Resúmen de características}
El dosímetros APS de \SI{4x4}{\micro\meter} que diseñamos, 
fabricamos y construímos presenta las
siguientes características:
\begin{itemize}
    \item Sensibilidad: \SI{7.1}{\volt\per\gray}
    \item Resolución: \SI{2.0}{\milli\gray}
    \item Rango (por disparo): \SI{0.4}{\gray}
\end{itemize}
%
\subsection{Conclusiones}
Presentamos el concepto básico del dosímetro APS,
con sus peculiaridades de uso y el tipo de mediciones que permite realizar.
Cubrimos su teoría de funcionamiento,
y de ahí explicamos el proceso de diseñar el circuito
para su fabricación.

En los resultados verificamos que el sensor
sigue el comportamiento básico que esperamos,
tanto en ausencia de radiación 
como para distintas intensidades de luz visible.
Asimismo, conseguimos una estimación inicial de la resolución del sensor
en base al ruido medido a la salida.
Así dimos los primeros pasos para implementar un dosímetro APS
en un proceso de fabricación CMOS estándar.
