\section{Floating Gate dosimeter}
Floating gate MOSFETs are used commercially for non-volatile memory (Flash and EEPROM).
The amount of charge placed on the gate represents the value of one or more bits of information.
This information can be retrieved by measuring the gate charge through the MOSFET's IV curves.

The floating gate dosimeter is based on a floating gate MOSFET.
In order to use it, it must first be reset by placing an initial charge on the gate.
The dosimeter will then integrate the total dose received from the moment it was reset.
This does not require a power supply during irradiation, but it is limited to a certain maximum dose.
This makes it ideal for long monitoring intervals,
for example during a person's workday.

We will follow the same outline as for the APS dosimeter,
starting with the theory behind its operation.
Afterwards, we will cover the design and optimization process.
Finally, we will present measurements of the fabricated sensor,
including irradiation results and the conclusions that follow.
%
\subsection{Operating principle}
The principle of operation of the FG dosimeter is that
incoming radiation which strikes the gate oxide
tends to discharge the gate.
Before irradiation,
the gate is charged using a tunneling current through the gate oxide
(\figref{fig:cargafg}).
\figp{cargafg}{figuras/fg/esquemainyeccion.pdf}
{ Charge injection in the FG using a tunnel current.
The voltage across the injector MOS translates to an electric field in the gate oxide,
which leads to a Fowler-Nordheim tunneling current.
The charge that enters the FG turns the reader MOS on.}
This charge alters the gate voltage, turning the reader MOS on.

Incoming radiation creates electron-hole pairs in the oxide that surrounds the floating gate.
This process causes a discharge
(\figref{fig:irradiacionfg}).
\figp{irradiacionfg}{figuras/fg/irradiacion.pdf}
{ FG discharge due to electron-hole pairs created by incoming radiation.
The holes go toward and recombine with the negative charge on the FG, thereby discharging it.}
The reader MOS then gradually turns off,
conducting less drain current for a given drain-source voltage
(\figref{fig:fg_vd_cte})
\fig{fg_vd_cte}{figuras/fg/lector_vd_cte.pdf}
{ Simulated drain current through the reader MOS as a function of gate voltage and $V_{sd}$.}
or increasing its drain-source voltage if the drain is biased at constant current
(\figref{fig:fg_id_cte}).
\fig{fg_id_cte}{figuras/fg/lector_id_cte.pdf}
{ Simulated reader drain voltage as a function of gate voltage and $I_d$.}
By calibrating one of these quantities against known doses,
one can build a dosimeter.
%
\subsection{Previous works}
Thomsen\cite{thomsen_floating-gate_1991} assembled a FG MOSFET
using a standard \SI{2}{\micro\meter} process with two polysilicon layers,
fabricated through the MOSIS service\cite{noauthor_mosis_nodate}.
Their innovation consists of using Fowler-Nordheim tunneling between the poly layers,
rather than hot electrons.
This enhances the tunneling current for both polarities,
achieving satisfactory injection using voltages under \SI{20}{\volt}.

Tarr\cite{tarr_sensitive_2004} fabricated a FG in a commercial
\SI{1.5}{\micro\meter} CMOS process with two poly layers.
This allows him to apply injection voltages up to \SI{40}{\volt}
without going through an injector MOS.
He used a reader MOS with an identical matched MOS,
in order to compensate for temperature drift.
He achieved a sensitivity of \SI{3}{\milli\volt\per\rad}.

Cesari\cite{cesari_floating_2014} fabricated a FG dosimeter
using a process with a single poly layer.
He used it to study the effect on the sensor of repeated charge-discharge cycles.
He charged and discharged the FG electrically by applying voltages up to
\SI{18}{\volt}.

%
\subsection{Capacitive coupling}
Because the FG is isolated from other nodes,
its voltage is a function of its charge and of the voltage of other nodes that are capacitively coupled to it.
Adding up the charges of all the capacitances connected to the FG leads to the equation
\begin{align}
    Q_{FG} &= (C_R + C_W) V_{FG} + C_I (V_{FG}-V_I)
    \label{eq:ccoupling}
\end{align}
with $Q_{FG}$ the FG charge, $C_R$ the reader MOS gate capacitance,
$C_W$ the FG capacitance over the reader N-well and 
$C_I$ the injector MOS gate capacitance (\figref{fig:cargadofg}).
\fig{cargadofg}{esquematicos/cargado_fg/cargado_fg.pdf}
{Schematic of the current flowing to the floating gate through the injector MOS gate oxide.
When the injector is biased, part of the applied voltage drops across its gate capacitance $C_I$,
and the rest drops across the N-well capacitance $C_W$ and reader MOS capacitance $C_R$.
By minimizing $C_I$, one maximizes the voltage across the injector,
thus maximizing the tunnel current.}
\subsection{Sensitivity}
In order to predict the sensitivity of the dosimeter,
we need to model the discharge of the floating gate
as a result of incoming radiation.

In principle, radiation generates electron-hole pairs in any volume of oxide it traverses.
We are only interested in those which are generated near the FG and are therefore capable of discharging it.
Therefore, under irradiation, each part of the FG contributes an ammount of charge proportional to its area $A$ and oxide thickness $t$.

As the FG is isolated, we can not directly measure its stored charge.
We can only measure it indirectly, through the gate voltage's effect on the reader MOS.
The ratio between FG charge and voltage is the total FG capacitance.
Each region of FG contributes a capacitance directly proportional to its area $A$ 
and inversely proportional to the thickness $t$ of the oxide 
separating it from the other plate of the capacitor 
(the N-well or substrate underneath).

Therefore, the sensitivity (the derivative of the gate voltage with respect to dose)
is the ratio of the charge generated in the oxide surrounding the FG,
to the capacitance from the FG to other nodes:
\begin{align}
    S = \deriv{V}{E} \propto \frac{\sum_i A_it_i}{\sum_j A_j/t_j}
    \label{eq:sensibilidad_fg}
\end{align},
with $A_i$ and $t_i$ the areas and thicknesses of the oxides surrounding the FG.
Both field oxide and gate oxide regions are made of SiO$_2$,
so the permittivity can be absorbed into the constant of proportionality.
By analyzing this equation, we can design the dosimeter in an optimal way which
maximizes sensitivity.
\subsection{Design}
The dosimeter's performance depends on the ratios between 
capacitances from the FG to other circuit nodes.
As the thickness of the dielectrics is not a design variable
(being determined by the fabrication process),
we can manipulate these ratios of capacitances through the
areas of the reader, injector and FG.
Due to limitations of the fabrication process (specifically lithography),
these areas have minimum values, and vary in discrete steps.
Additionally, we have a limited total area available.
In order to choose an optimal set of areas,
we explored the problem space with two benchmarks in mind:
radiation sensitivity, and injection efficiency.

Equation~\ref{eq:sensibilidad_fg} implies that sensitivity is maximized
by giving the most area to the thickest oxide,
which is the field oxide between the FG and the reader N-well.

On the other hand, equation~\ref{eq:ccoupling}
says that the tunnel voltage $V_{FG}-V_I$
is maximized, for a given $V_I$,
by minimizing the $C_I/C_{FG}$ ratio.
Because there is a lower limit to the injector area,
it is necessary to increase the other areas to reduce that ratio.

We explored the problem space by plotting contours for both benchmarks,
as a function of two design variables:
reader area / injector area,
and reader N-well area / injector area.
These contour plots are shown in figures~\ref{fig:sensibilidad_fg}
and~\ref{fig:eficiencia_inyeccion}.
\fig{sensibilidad_fg}{figuras/fg/sensibilidad.pdf}
{Floating gate sensitivity 
as a function of the ratios between
injector area ($A_I$),
reader area ($A_R$) 
and reader N-well area ($A_W$).}
\fig{eficiencia_inyeccion}{figuras/fg/inyeccion.pdf}
{Fraction of the injection voltage which drops across the injector gate oxide,
as a function of the ratios between
injector area ($A_I$),
reader area ($A_R$) 
and reader N-well area ($A_W$).}

% TODO Detallar criterios utilizados para llegar a esos números.
% Las relaciones Aw/Ai y Ar/Ai están fuera de los gráficos anteriores
These criteria lead us to use an injector with minimum area.
This leaves the reader and N-well areas as free variables.
\Figref{fig:eficiencia_inyeccion}
shows that a $A_R/A_I$ ratio around 10000
leads to an injection efficiency of approximately 1.
By filling the rest of the available area with reader N-well,
we get a $A_W/A_I$ ratio of almost 40000.
\Figref{fig:sensibilidad_fg} shows that this choice of areas
yields a sensitivity in the tens of \SI{}{\milli\volt\per\gray},
which is satisfactory.
After laying out the structures in the area allocated for this project,
we reached the final areas for each region:
\begin{table}[h]
\centering
\begin{tabular}{|c|c|}
\hline
Region   & Area (\SI{}{\micro\meter\squared})\\ \hline
Injector & 4.32\\
N-Well     & 180000\\
Reader   & 35000\\
\hline
\end{tabular}
\end{table}
%
\subsection{Physical design (layout)}
%
\fig{layout_fg_todo}{figuras/gds/fg/small/poly_met.png}
{Layout of the complete dosimeter, showing polysilicon and metal layers.
The upper polysilicon rectangle is laid over N-well with field oxide,
and is the main charge-generating region due to its thick oxide.
On the lower left is the injector,
a minimum-area MOS through which the FG is charged.
On the lower right is the reader MOS, laid out with multiple fingers
(parallel MOSFETs which share drain/source implants).}
The layout (\figref{fig:layout_fg_todo}) is divided into 3 main regions:
\begin{itemize}
    \item Floating Gate over field oxide and N-well,
    \item MOS injector, and
    \item reader MOSFET.
\end{itemize}
The injector is a minimum-area MOS in a separate N-well
(\figref{fig:layout_inyector}).
\fig{layout_inyector}{figuras/gds/fg/small/inyector.png}
{Injector MOS layout. It is surrounded by body contacts.
These are shorted with the drain/source contacts
in a metal layer that is not shown.  }
This allows the drain, source and body terminals to be connected to the injection terminal of the chip. If the N-well were shared, it would not be possible to apply high voltages to it without disrupting other parts of the circuit.

The reader is a $W=\SI{100}{\micro\meter}$ 
by $L=\SI{25}{\micro\meter}$ MOSFET with 14 fingers.
This means that there are 14 MOSFETs in parallel,
sharing source/drain implants. 
This is visible in the metal layers (\figref{fig:layout_fg_met}),
where the sources are joined on the bottom using M1 (the first metal layer)
and the drains are joined on top using M2 (the second metal layer).
\fig{layout_fg_met}{figuras/gds/fg/small/met1_met2.png}
{FG metal layers.
On the left is M1 (first metal layer)
shorting source, drain and body of the injector MOS.
On the bottom right is M1 connecting the sources and body contacts of the reader MOS.
Above that is M2 joining the reader MOS drains.}
\subsection{Charge measurement}
\fig{floatingcapacidades}{figuras/fgcapacidades/floatinggate2.pdf}
{Capacitive coupling model in a FGMOS}
The floating gate voltage controls the level of inversion
in a MOSFET we will call "reader".
In order to find the channel inversion as a function of FG charge,
we analyse the capacitive coupling between the FG and other nodes
\cita{pavan_floating_2004}.
Solving equation~\ref{eq:ccoupling}, we reach
\begin{align*}
    V_{FG} &= \frac{C_I V_I + (C_R+C_W) V_R + Q}{C_I+C_R+C_W}
\end{align*}
with the terms shown in \figref{fig:floatingcapacidades}.

While reading, the circuit is biased at $V_I=V_R=0$.
We can evaluate the equations in section~\ref{section:ecuaciones_mos}
(for a PMOS) and write them as a function of $V_{FG}$ and $V'_R$.
This way we reach an expression for the reader current
\begin{align*}
    I_R' &= \begin{cases}
        I_{D0} \left(\frac W L\right)_L
        \exp\left(\frac{V_{FG}-V_T}{nkT/q}\right)& V_{FG}>-V_T\\
        \beta_n\left(\frac W L\right)_L(V_{FG}+V_T+\frac{V'_R}2)V'_R &
        -V'_R-V_T<V_{FG}<-V_T\\
        \frac{\beta_n}2\left(\frac W L\right)_L(V_{FG}+V_T)^2 &
        -V'_R-V_T>V_{FG}.
    \end{cases}
\end{align*}
. These equations tell us that,
when the reader is biased with a small
$V'_R$ (we used \SI{-0.1}{\volt}),
we are in the second case and the drain current
is a linear function of $V_{FG}$.
\subsection{Charging the floating gate}
%
\subsubsection{Injection mechanism}
Because the floating gate is surrounded by insulators,
it is not possible to charge it or discharge it through normal conduction
like a regular capacitor.
The charge must go through the surrounding insulation.

Normally, insulators have very few charge carriers.
This prevents the kind of conduction seen in metals.
In addition, the insulator is usually thick enough
to prevent electrons from tunneling directly across the potential barrier.
Therefore, the resistance is very high.

When a high enough voltage is applied,
the electric field within the insulator tilts the conduction band,
narrowing the potential barrier
(\figref{fig:fowlernordheim}).
This increases the tunnel probability and, therefore, the current.
This is called a
Fowler-Nordheim tunneling current\cite{lenzlinger_fowlernordheim_1969}.
\fig{fowlernordheim}{figuras/fowlernordheim/fowlernordheim.pdf}
{Band diagram for electron emission from the channel to the gate of a MOS.
The electric field in the gate oxide narrows the oxide potential barrier,
making tunneling more probable.
Reprinted from \cite{lenzlinger_fowlernordheim_1969}}

This current fits an expression of the form
\begin{align*}
    J_{FN} &= AF_{ox}^2\exp(-B/F_{ox}).
\end{align*}
with two fitting constants $A$ y $B$,
and $F_{ox}$ the oxide electric field.

This explains the current that flows through a MOS' gate
when a high enough voltage is applied between gate and body.
In our case, we can't apply a voltage directly to the gate because it is floating.
This can be seen in \figref{fig:cargadofg}.
We apply a voltage to the body of an injector MOS,
whose gate is the FG.
By minimizing the area of this MOS,
we lower its gate capacitance.
This way, most of the applied voltage drops across its gate oxide,
causing a tunneling current which charges/discharges the FG.

Because we want the FG to turn on a PMOS,
we need to give it a negative charge.
Looking at \figref{fig:cargadofg},
a negative voltage between injector and reader
stores a negative charge in the FG.
This biases the injector MOS in accumulation,
so its gate-body voltage drops mainly across the gate oxide
(and to a much lesser extend across the silicon).
On the other hand, almost no voltage drops across the reader MOS,
because its capacitance is much greater than that of the injector.

% FIXME: esto queda colgado
The oxide field $F_{ox}$ can be found from the MOS capacitor equations.
The injector's gate voltage is $V_{FG}$
and its body voltage is $V_I$.
Substituting these values into the MOS charge balance equation
(equation~\ref{eq:potencial_campo_mos})
one reaches the expression
\begin{align*}
    V_{FG}-V_I &= F_{ox}t_{ox}+\psi_s+V_{FB},
\end{align*}
with $V_{FB}=(\Phi_S-\Phi_M)/e$ 
and $\psi_s$ the voltage drop across the injector silicon,
which is, as we said, negligible.
%
%Nuestro proceso de fabricación alcanza breakdown del
%óxido de gate al aplicar \SI{13}{\volt}.
%A esta tensión la densidad de corriente es
%\SI{.1}{\nano\ampere\per\micro\meter\squared}, cargando nuestro floating
%gate a razón de \SI{3.9}{\volt\per\second}.
% TODO: calcular / medir curva Fowler-Nordheim
\subsubsection{Measurement procedure}
\fig{medicion_fg}{esquematicos/medicion_fg/medicion_fg.pdf}
{Experimental setup for injecting current into the FG,
including all relevant leakage paths.
By connecting the substrate to the current source guard terminal,
the voltage across the injector's body-substrate diode is made 0.}
We charged the floating gate by applying a constant current 
between the injector and the reader N-well.
While charging, any parasitic conduction path between those nodes
will carry part of the injection current,
reducing the injected charge.
Moreover, part of the current serves to charge the system capacitances.
If the applied current is small (in order to charge the FG slowly),
the setup spends most of its time charging capacitances,
until they reach a high enough voltage to produce tunneling.

We used the current source's guard terminal to zero the voltage across some of the leakage paths
(\figref{fig:medicion_fg}).
By connecting it to the substrate,
we eliminate the reverse current through the injector's bulk-substrate diode.
Since the injector was bonded to the pin adjacent to that of the reader N-well,
it was not possible to connect the guard between those two pins 
to prevent leakage through the PCB.
\subsubsection{Charge/discharge curves}
The injector voltage (figures~\ref{fig:descarga_inyector}
and~\ref{fig:carga_inyector})
changes linearly as the cable capacitance charges at constant current.
When the potential difference between injector and FG is high enough,
the tunnel current increases rapidly, and the voltage rate of change decreases.

In order to study the FG charge and discharge process,
we repeated this sequence multiple times:
\begin{itemize}
    \item Force a constant current into the injector,
        while biasing the reader at $V_{sd}=$\SI{100}{\milli\volt}
        and measuring the reader current.
        The injection is stopped when the reader current changes by a predetermined amount.
    \item Measure reader current as a function of $V_{sd}$.
        On every iteration, this curve is higher/lower when charging/discharging, respectively.
\end{itemize}

The IV curves (figures~\ref{fig:descarga_iv}
and~\ref{fig:carga_iv}) have increasing/decreasing saturation currents,
which confirms that the FG is charging/discharging
between measurements.
\fig{descarga_inyector}{figuras/fg/21a29dip_inyector.pdf}
{FG discharge measuring injector voltage (dotted line) and
reader drain current (solid line) at
$V_{sd}$=\SI{100}{\milli\volt}.
This current is an indicator of the amount of charge in the FG.}
\fig{descarga_iv}{figuras/fg/21a29dip_iv.pdf}
{Reader IV curves, measured between discharge intervals of
    \figref{fig:descarga_inyector}.}
\fig{carga_inyector}{figuras/fg/12a21dip_inyector.pdf}
{FG charging, measuring injector voltage (dotted line) and
reader drain current (solid line) at
$V_{sd}$=\SI{100}{\milli\volt}.}
\fig{carga_iv}{figuras/fg/12a21dip_iv.pdf}
{Reader IV curves, measured between charging intervals of
    \figref{fig:carga_inyector}.}
\subsection{\Strontium irradiation}
We charged the dosimeter and exposed it to a 
\Strontium source
in the irradiator described in chapter~\ref{sec:irradiador}.
\Figref{fig:irradiacionfg_respuesta} shows that the current
varies almost linearly with the dose.
The sensitivity (slope of the current vs dose curve)
can be seen more clearly in
\figref{fig:irradiacionfg_sensibilidad}.

We fit the measurements to a simple model.
The model extracts the initial FG charge from the initial measured current.
We use the initial slope of the current to extract the charge generated per unit dose.
Afterwards, we calculate the discharge of the FG by numerically integrating the charge generated by radiation.
We go from FG charge to FG voltage using the capacitive coupling equation.
Finally, we get the reader current by using the reader transistor's simulated IV curves.

The strictly electrical part of this calculation 
(dealing with charges, capacitances, IV curves)
is based on process parameters that are tightly controlled.
In addition, the components involved have large areas, which reduces the statistical variation of their characteristics.
This gives us confidence in this part of the model.

However, the simulated sensitivity is consistently higher than what was measured.
This is similar to a problem that is known when working with MOS radiation effects,
when one assumes that all of the oxide generated charge is near the interface.
From equation~\ref{eq:corrimiento_vt_mos}, this is where said charge causes the greatest $V_t$ shift.
In our case, we are not dealing with oxide trapped charge, but rather with charge that is accumulated on a gate.
This suggests that there are other processes taking place during irradiation, which lower the sensitivity.
For instance, it is possible that there is a $V_t$ shift due to Radiation Induced Charge Neutralization
\cite{faigon_extension_2008}\cite{fleetwood_radiation-induced_1990}.
Future works could take these effects into account in order to fully understand the dosimeter's behavior.

\fig{irradiacionfg_respuesta}{figuras/fg/irradiacion_corriente.pdf}
{Current through the FG reader biased at 
    $V_{sd}$=\SI{100}{\milli\volt}, as a function of dose.
The calculated current starts from the initial measured current.
The charge generated per unit dose is extracted from the initial measured sensitivity.
The FG charge is calculated by numerically integrating the radiation generated charge.
This charge is translated into a voltage using the capacitive coupling equation.
Finally, the reader current is interpolated from the FG voltage,
using the reader MOS' simulated IV curves.}
\fig{irradiacionfg_sensibilidad}{figuras/fg/irradiacion_sensibilidad.pdf}
{Sensitivity of the FG biased at 
    $V_{sd}$=\SI{100}{\milli\volt}, as a function of dose.
}
% FIXME: Tanto en este gráfico como en el de arriba habría que comentar -aparte
% de lo puesto en la discusión- algo sobre las dos curvas: su parecido, su
% falta de parecido\dots o sacar la calculada.
% Si no es mucho trabajo graficá el módulo en el eje Y. De modo que el
% crecimiento de la sensibilidad se vea en un tramo que apunta hacia arriba y
% no hacia abajo.
The sensitivity varies with dose, due to two effects.
\begin{itemize}
    \item As the FG discharges, the oxide field diminishes.
        This reduces the electron-hole pair production yield.
    \item As the current goes down, $g_m$ (the transconductance, defined as $\frac{dI_D}{dV_G}$) goes up
        (\figref{fig:didv}), which increases the sensitivity.
\end{itemize}
Measurements show a growth in sensitivity until the dose reaches
\SI{50}{\Gray}.
\fig{didv}{figuras/fg/didv.pdf}
{The slope of the reader $I_D(V_G)$ curve (\figref{fig:fg_vd_cte}) 
increases as $I_D$ drops down to 
$\approx$\SI{50}{\micro\ampere}.
This increases the FG sensitivity.
This curve was simulated using foundry models,
as the gate of the actual fabricated device is inaccessible.
}
\subsection{Noise current}
We defined a measurement of this dosimeter as the average of 10 current samples,
spaced \SI{1}{\second} apart.
Using this definition, we can define the noise as the standard deviation of this average.

As with the APS, starting from this definition precludes a more general study
of the sensor's intrinsic noise characteristics.
Instead, we focus on the noise for this specific amount of averaging and sampling rate.
A more general analysis would study the noise power spectral density,
allowing for different noise vs measurement time tradeoffs.

We measured the reader current without radiation,
and calculated the difference between adjacent readings in order to reduce the effect of drift
(\figref{fig:ruidofg}).
% TODO Al igual que en el APS habría que precisar qué representa este
% sigma... una estimación del ruido a qué frecuencia?
This gives us a standard deviation of
$\sigma=$ \SI{27}{\nano\ampere},
which corresponds to a dose of \SI{4}{\milli\gray}.
This means that, when the user makes a reading,
the difference between the result and the real dose
will have a random component whose standard deviation is
\SI{4}{\milli\gray}.
\fig{ruidofg}{figuras/fg/ruido.pdf}{
    Difference between adjacent reader current readings,
scaled to show the noise when averaging 10 samples.}
\subsection{Summarized specifications}
The FG dosimeter that we've designed and built has the following specifications:
\begin{itemize}
    \item Sensitivity: between 5.5 y \SI{7}{\micro\ampere\per\gray}
    \item Resolution: \SI{4}{\milli\gray}
    \item Range: \SI{100}{\gray}
    \item Initial voltage: up to $\pm$\SI{16}{\volt}
\end{itemize}.
