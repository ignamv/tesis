\section{Dosímetro Floating Gate}
Los MOSFET con gate aislado o floating gate se usan comercialmente
en memorias no volátiles (Flash y EEPROM).
Se coloca cierta carga, 
cuyo valor representa uno o más bits de información, en el gate.
Luego se recupera la información midiendo la carga del gate
en base a las curvas IV del MOSFET.

El dosímetro FG se basa en un MOSFET con gate aislado.
Su modo de uso es primero resetearlo, dándole una carga inicial.
Desde el momento de reset, el dosímetro integra 
la dosis total que recibe sin necesidad de alimentación
(mientras no exceda una dosis máxima).
Por lo tanto es idóneo para seguimientos largos,
por ejemplo de un trabajador durante su turno.

Igual que como hicimos para el dosímetro APS,
esta parte del trabajo comienza con la teoría del dosímetro.
Luego cubre cómo realizamos el diseño y lo optimizamos.
Por último, presenta las mediciones del sensor fabricado,
incluyendo los resultados de su irradiación,
y las conclusiones que siguen de ellas.
%
\subsection{Principio de funcionamiento}
El principio de funcionamiento del dosímetro FG 
es que la radiación que incide
en el óxido de compuerta tiende a descargar al FG.
Antes de irradiar,
se coloca carga en el gate mediante una corriente de túnel 
(\figref{fig:cargafg}).
\figp{cargafg}{figuras/fg/esquemainyeccion.pdf}
{ Inyección de carga en el FG a través de una corriente de túnel.
La tensión en el inyector produce un campo eléctrico en su óxido de gate,
que facilita una corriente de túnel Fowler-Nordheim.
La carga que pasa al FG prende el transistor lector.}
Esta carga altera la tensión de gate, prendiendo un transistor lector.

La radiación incidente produce pares electrón-hueco en el óxido
que rodea al floating gate, descargándolo
(\figref{fig:irradiacionfg}).
\figp{irradiacionfg}{figuras/fg/irradiacion.pdf}
{ Descarga del FG debido a pares electrón-hueco creados por
radiación.
Los huecos son atraídos a la carga negativa del FG y
se recombinan con la misma, descargándolo.}
Esto va apagando el lector,
reduciendo su corriente de drain si se aplica tensión drain-source constante
(\figref{fig:fg_vd_cte})
\figp{fg_vd_cte}{figuras/fg/lector_vd_cte.pdf}
{Variación de la corriente de drain del lector con la tensión de gate,
para distintos valores de $V_{sd}$.}
o aumentando su tensión drain-source si se aplica corriente de drain constante
(\figref{fig:fg_id_cte}).
\figp{fg_id_cte}{figuras/fg/lector_id_cte.pdf}
{Variación de la tensión de drain del lector con la tensión de gate,
para distintos valores de $I_d$.}
Calibrando estas cantidades contra la dosis recibida, 
el resultado es un dosímetro.
%
\subsection{Trabajos previos}
Thomsen\cite{thomsen_floating-gate_1991} produce un FG MOSFET
con un proceso estándar de \SI{2}{\micro\meter} con dos capas de polisilicio
a través del consorcio MOSIS\cite{noauthor_mosis_nodate}.
Su innovación consiste en usar túnel Fowler-Nordheim entre las capas de
polisilicio, en vez de usar electrones calientes. 
Así logra buenas corrientes de inyección para ambas polaridades,
aplicando tensiones de hasta \SI{20}{\volt}.

Tarr\cite{tarr_sensitive_2004} fabrica un FG en un proceso comercial CMOS
de \SI{1.5}{\micro\meter} con dos capas de polisilicio. 
Esto le permite aplicar la tensión de inyección 
(de hasta 40V) sin pasar por un MOS.
Usa un MOS lector apareado con otro MOS idéntico para compensar 
la variación con temperatura.
Como resultado obtiene una sensibilidad de \SI{3}{\milli\volt\per\rad}.

Cesari\cite{cesari_floating_2014} construye un dosímetro FG 
con un proceso de una sola capa de polisilicio.
Lo usa para estudiar el efecto de cargas y descargas repetidas
en el sensor.
En ambos casos controla la carga del FG eléctricamente,
aplicando tensiones de hasta \SI{18}{\volt}.

%
\subsection{Acoplamiento capacitivo}
Ya que el FG está aislado,
su tensión es función de la carga que almacena 
y de la capacidad entre este y otros nodos del circuito.
Sumando la carga en cada capacidad se llega a la relación
\begin{align}
    Q_{FG} &= (C_R + C_W) V_{FG} + C_I (V_{FG}-V_I)
    \label{eq:ccoupling}
\end{align}
con $Q_{FG}$ la carga almacenada en el FG, $C_R$ la capacidad de gate del
lector, $C_W$ la capacidad del FG sobre el well del lector y $C_I$ la capacidad
de gate del inyector.
\subsection{Sensibilidad}
Para predecir la sensibilidad del dosímetro,
necesitamos un modelo simple de cómo la radiación
descarga al floating gate.

En principio, la radiación genera pares electrón-hueco
en todo el volúmen de óxido donde incide.
Esto sólo nos interesa cuando ocurre en el óxido que rodea al FG,
porque entonces esos pares electrón-hueco pueden descargarlo.
Por lo tanto, cada región de FG aporta una generación de carga
proporcional a su área $A$ y al espesor de óxido: $t$.

No podemos medir directamente la cantidad de carga en el floating gate.
Sólo podemos saberla indirectamente, a partir de la tensión de gate.
La relación entre carga y tensión es la capacitancia del FG.
Para cada región de FG, su capacitancia es proporcional al área $A$
e inversamente proporcional al espesor del óxido $t$
que la separa de la otra placa del capacitor (lo que sea que tenga debajo).

Por lo tanto, la sensibilidad 
(definida como derivada de la tensión respecto de la dosis)
es un cociente entre la carga generada por todo el óxido que rodea al FG
y la capacitancia del FG a otros nodos.
\begin{align}
    S = \deriv{V}{E} \propto \frac{\sum_i A_it_i}{\sum_j A_j/t_j}
    \label{eq:sensibilidad_fg}
\end{align},
con $A_i$ y $t_i$ las áreas y espesores de los óxidos que rodean al FG.
Tanto los óxidos de campo como de gate consisten de SiO$_2$,
de modo que la permitividad sale como constante.
Explorando esta ecuación, podemos optimizar el diseño del dosímetro
para maximizar su sensibilidad.
\subsection{Diseño}
El desempeño del dosímetro depende de cocientes
de capacidades entre FG y distintos nodos del circuito. 
Estos cocientes se reducen a relaciones entre las áreas del lector,
inyector y FG.
Debido a limitaciones del proceso de fabricación,
las áreas tienen un valor mínimo y varían de a pasos discretos.
Por otro lado, hay un área total máxima para el dosímetro.
Para elegir las dimensiones óptimas,
exploramos el espacio de soluciones mirando dos parámetros de calidad:
sensibilidad a la radiación, y eficiencia de inyección.

La ecuación~\ref{eq:sensibilidad_fg} dice que la sensibilidad se maximiza
asignando mayor área al óxido más grueso,
que es el óxido de campo entre FG y el well del lector.

Por otro lado,
la ecuación~\ref{eq:ccoupling} dice que la tensión de túnel $V_{FG}-V_I$
se maximiza, para un $V_I$ dado,
minimizando la relación $C_I/C_{FG}$.
Dado que hay un área mínima para el inyector,
es necesario aumentar las otras capacidades para reducir esa relación.

Exploramos el espacio de soluciones
graficando las curvas de nivel de ambos parámetros en función de dos variables
de diseño:
la relación área lector / área inyector,
y la relación área de well del lector / área inyector.
Esto se ve en las figuras~\ref{fig:sensibilidad_fg}
y~\ref{fig:eficiencia_inyeccion}.
\fig{sensibilidad_fg}{figuras/fg/sensibilidad.pdf}
{Sensibilidad del floating gate en función de la relación de áreas de 
inyector ($A_I$),
lector ($A_R$) 
y well del lector ($A_W$).}
\fig{eficiencia_inyeccion}{figuras/fg/inyeccion.pdf}
{Fracción de la tensión de inyección que cae en el óxido del inyector,
en función de la relación de áreas de 
inyector ($A_I$),
lector ($A_R$) 
y well del lector ($A_W$).}

% TODO Detallar criterios utilizados para llegar a esos números.
% Las relaciones Aw/Ai y Ar/Ai están fuera de los gráficos anteriores
El inyector, por las razones ya explicadas, es de área mínima.
Quedan por determinar las dimensiones finales del lector y el inyector.
La \figref{fig:eficiencia_inyeccion}
muestra que con una relación $A_R/A_I$ de cerca de 10000
se obtiene una eficiencia de inyección casi igual a 1.
Rellenando el resto del área disponible en el chip con Well del lector,
obtenemos una relación $A_W/A_I$ de casi 40000.
La \figref{fig:sensibilidad_fg} muestra que con estas áreas
se obtiene una sensibilidad de varias decenas de \SI{}{\milli\volt\per\gray},
lo cual resulta aceptable.
Ajustando las áreas para que quepa todo 
en la región reservada para este proyecto 
llegamos a las dimensiones finales de cada región:
\begin{table}[h]
\centering
\begin{tabular}{|c|c|}
    \hline
    Región   & Área (\SI{}{\micro\meter\squared})\\ \hline
Inyector & 4.32\\
Well     & 180000\\
Lector   & 35000\\
\hline
\end{tabular}
\end{table}
%
\subsection{Diseño físico (layout)}
%
\fig{layout_fg_todo}{figuras/gds/fg/small/poly_met.png}
{Layout del dosímetro completo, mostrando polisilicio y metalización.
El rectángulo superior de polisilicio está sobre nwell con óxido de campo,
y es la región principal de generación de carga por su óxido grueso.
Abajo a la izquierda está el inyector, 
el MOS de área mínima a través del cual se carga al FG.
Abajo a la derecha está el transistor lector,
armado con múltiples canales 
(varios MOSFET en paralelo que comparten difusiones source/drain).}
El layout (\figref{fig:layout_fg_todo}) se divide en 3 grandes regiones:
\begin{itemize}
    \item Floating Gate sobre óxido de campo y nwell,
    \item MOS inyector, y
    \item MOSFET lector.
\end{itemize}
El inyector es un MOS de área mínima con su propio nwell
(\figref{fig:layout_inyector}).
\fig{layout_inyector}{figuras/gds/fg/small/inyector.png}
{Layout del MOS inyector. Está rodeado por contactos a body (nwell).
Estos contactos y los de drain/source 
están cortocircuitados por la metalización (no visible).}
Esto permite conectar su drain, source y body a la terminal de inyección.

El lector es un MOSFET de $W=\SI{100}{\micro\meter}$ 
por $L=\SI{25}{\micro\meter}$ con 14 canales.
Esto significa que usa 14 MOSFET en paralelo,
que comparten difusiones source/drain. 
Esto se ve claramente en la metalización (\figref{fig:layout_fg_met}),
donde los source están conectados por debajo con M1 (la primera capa de metal)
y los drain por arriba con M2 (la segunda capa de metal).
\fig{layout_fg_met}{figuras/gds/fg/small/met1_met2.png}
{Metalización del FG.
A la izquierda, hay M1 (la primera capa de metal) 
cortocircuitando source, drain y body del MOS inyector.
A la derecha, hay M1 conectando por debajo los sources/body del MOSFET lector
y M2 conectando por arriba los drains.}
\subsection{Medición de la carga}
\fig{floatingcapacidades}{figuras/fgcapacidades/floatinggate2.pdf}{Modelo de
acoplamiento capacitivo en un MOSFET con floating gate.}
La tensión del floating gate controla el canal de un MOSFET
que llamamos lector o reader.
Para determinar esa tensión en función de la carga del FG,
usamos el acoplamiento capacitivo
\cita{pavan_floating_2004} 
entre floating gate y otros nodos, llegando a la ecuación
\begin{align*}
    V_{FG} &= \frac{C_I V_I + (C_R+C_W) V_R + Q}{C_I+C_R+C_W}
\end{align*}
con los términos ilustrados en la \figref{fig:floatingcapacidades}.

Durante la lectura se usa $V_I=V_R=0$.
En función de $V_{FG}$ y $V'_R$, la corriente del lector es
\begin{align*}
    I_R' &= \begin{cases}
        I_{D0} \left(\frac W L\right)_L
        \exp\left(\frac{V_{FG}-V_T}{nkT/q}\right)& V_{FG}>-V_T\\
        \beta_n\left(\frac W L\right)_L(V_{FG}+V_T+\frac{V'_R}2)V'_R &
        -V'_R-V_T<V_{FG}<-V_T\\
        \frac{\beta_n}2\left(\frac W L\right)_L(V_{FG}+V_T)^2 &
        -V'_R-V_T>V_{FG}.
    \end{cases}
\end{align*}
Estas ecuaciones nos indican que,
polarizando el lector con una tensión $V'_R$ pequeña
(usamos \SI{-0.1}{\volt}),
estamos en la situación del medio y
la corriente de drain varía linealmente con $V_{FG}$
\subsection{Cargado del floating gate}
%
\subsubsection{Mecanismo de inyección}
Dado que el floating gate está aislado,
no es posible cargarlo o descargarlo con una fuente de tensión
como las placas de un capacitor normal.
La única forma de hacerle llegar carga es 
a través del aislante que lo rodea.

En condiciones normales, el aislante tiene muy pocos portadores.
Esto impide la conducción normal como en un metal.
Además, para espesores típicos de aislante, 
los electrones que están a uno y otro lado del mismo
no pueden tunelear a través de su barrera de potencial.
En consecuencia, el aislante presenta una resistencia alta.

Al aplicarle suficiente tensión,
el campo eléctrico en el aislante inclina la banda de conducción,
reduciendo el ancho de la barrera de potencial
(\figref{fig:fowlernordheim}).
Así aumenta la probabilidad de túnel, y en consecuencia la corriente.
Esto se denomina corriente de túnel
Fowler-Nordheim\cite{lenzlinger_fowlernordheim_1969}.
\fig{fowlernordheim}{figuras/fowlernordheim/fowlernordheim.pdf}
{Diagrama de bandas de la emisión de electrones del canal al gate de un MOS. 
El campo eléctrico en el óxido de gate reduce el ancho de la barrera de
potencial del óxido, facilitando el tuneleo.
Reproducido de \cite{lenzlinger_fowlernordheim_1969}}

Esta corriente se ajusta a una expresión del tipo
\begin{align*}
    J_{FN} &= AF_{ox}^2\exp(-B/F_{ox}).
\end{align*}
con $A$ y $B$ constantes de ajuste 
y $F_{ox}$ el campo eléctrico en el óxido.

Esto explica la corriente de gate que fluye en un MOS,
al aplicar tensión suficiente entre body y gate.
En nuestro caso, no podemos aplicar tensión directamente al gate
porque está completamente aislado.
Esto puede representarse como en la \figref{fig:cargadofg}.
\fig{cargadofg}{esquematicos/cargado_fg/cargado_fg.pdf}
{Esquemático del flujo de carga al floating gate 
a través del óxido del MOS inyector.
Al aplicar tensión al inyector,
parte de esta tensión cae sobre 
la capacidad de gate $C_I$ del MOS inyector 
y parte sobre la capacidad de well $C_W$ y del MOS lector $C_R$.
Minimizando $C_I$, se maximiza la tensión a través del inyector
y por lo tanto se logra la mayor corriente de túnel.}
La tensión se aplica al body de un MOS inyector,
cuyo gate es el FG.
Dándole al MOS inyector la menor área posible, 
reducimos su capacidad de gate.
Así la mayor parte de la tensión de inyección cae a través de su óxido
y produce una corriente de túnel que carga al FG.

Dado que el FG tiene que prender un PMOS,
hay que darle una carga inicial negativa.
Mirando la \figref{fig:cargadofg},
una tensión negativa en el inyector
produce carga negativa en el FG.
Esta polarización pone al MOS inyector en acumulación,
de modo que la tensión gate-body cae principalmente 
en el óxido de gate (y muy poco en el silicio).

% FIXME: esto queda colgado
El campo en el óxido $F_{ox}$ sigue la expresión para el campo en el óxido
de gate de un MOS.
En el caso del inyector, la tensión de gate es $V_{FG}$
y la tensión de body es $V_I$.
Reemplazando estas dos cantidades en la ecuación de balance de carga
del MOS (ecuación~\ref{eq:potencial_campo_mos})
y agregando los términos debido a la no-idealidad de la estructura
se llega a la expresión
\begin{align*}
    V_{FG}-V_I &= F_{ox}t_{ox}+\psi_s+V_{FB},
\end{align*}
con $V_{FB}=(\Phi_S-\Phi_M)/e$ 
y $\psi_s$ la caída de tensión sobre el Si del inyector,
que como ya dijimos es despreciable.
%
%Nuestro proceso de fabricación alcanza breakdown del
%óxido de gate al aplicar \SI{13}{\volt}.
%A esta tensión la densidad de corriente es
%\SI{.1}{\nano\ampere\per\micro\meter\squared}, cargando nuestro floating
%gate a razón de \SI{3.9}{\volt\per\second}.
% TODO: calcular / medir curva Fowler-Nordheim
\subsubsection{Experimental}
\fig{medicion_fg}{esquematicos/medicion_fg/medicion_fg.pdf}
{Setup experimental para inyectar corriente en el FG,
con todos los caminos de pérdidas relevantes.
La conexión del sustrato a la guarda de la fuente de corriente
anula la tensión a través del diodo sustrato-bulk del inyector.}
Cargamos el floating gate aplicando corriente constante
entre el inyector y el well del lector.
Durante la inyección,
cualquier conductancia parásita entre esos nodos 
va a llevarse parte de la corriente,
reduciendo la carga inyectada.
Al mismo tiempo, parte de la carga proporcionada
va a cargar las capacidades del sistema.
Si aplicamos una corriente pequeña
(para cambiar lentamente la carga del FG),
el setup de inyección está la mayoría del tiempo cargando estas capacidades
hasta que se alcanza la tensión necesaria para el tuneleo de inyector a FG.

Usamos la guarda de la fuente de corriente para eliminar algunos caminos de pérdida
(\figref{fig:medicion_fg}).
Conectándola al sustrato evitamos la corriente en inversa del diodo de bulk
del inyector.
Ya que el inyector está bondeado al pin siguiente al well del lector,
no es posible interponer una guarda entre ellos para evitar pérdidas en el PCB.
\subsubsection{Curvas de carga y descarga}
La tensión del inyector (figuras~\ref{fig:descarga_inyector}
y~\ref{fig:carga_inyector})
varía linealmente mientras se carga la capacidad de los cables a corriente
constante.
Cuando alcanza una diferencia de potencial suficiente respecto del FG,
la corriente de túnel crece rápidamente y frena el crecimiento de la tensión.
Las curvas IV (figuras~\ref{fig:descarga_iv}
y~\ref{fig:carga_iv}) saturan a corrientes cada vez más grandes/chicas,
confirmando la variación de carga del FG entre cada tramo
de inyección.
\fig{descarga_inyector}{figuras/fg/21a29dip_inyector.pdf}
{Descarga del FG midiendo tensión del inyector (línea punteada) y
corriente de drain del lector (línea sólida) a
$V_{sd}$=\SI{100}{\milli\volt}.
Esta corriente de drain es una indicación directa 
de la cantidad de carga en el FG.}
\fig{descarga_iv}{figuras/fg/21a29dip_iv.pdf}
{Curvas IV del lector medidas entre tramos de la descarga
    de la \figref{fig:descarga_inyector}.}
\fig{carga_inyector}{figuras/fg/12a21dip_inyector.pdf}
{Carga del FG midiendo tensión del inyector (línea punteada) y
corriente de drain (línea sólida) a
$V_{sd}$=\SI{100}{\milli\volt}.}
\fig{carga_iv}{figuras/fg/12a21dip_iv.pdf}
{Curvas IV del lector medidas entre tramos de la carga.}
\subsection{Irradiación con \Strontium}
Expusimos el dosímetro,
previamente cargado,
a una fuente de \Strontium.
La \figref{fig:irradiacionfg_respuesta} muestra que la corriente responde de
manera casi lineal a la dosis.
La variación de la sensibilidad se ve con más detalle en la
\figref{fig:irradiacionfg_sensibilidad}.
\fig{irradiacionfg_respuesta}{figuras/fg/irradiacion_corriente.pdf}
{Corriente del lector del FG polarizado con
    $V_{sd}$=\SI{100}{\milli\volt} en función de la dosis recibida.
La corriente calculada parte de la corriente inicial extraída de la
medición.
Primero se calcula la variación de carga del FG 
integrando numéricamente la carga generada por radiación.
Luego se traduce esa carga en una tensión
usando la ecuación de acoplamiento capacitivo.
Finalmente se obtiene la corriente del lector
en función de la tensión del FG,
usando las curvas simuladas del transistor lector.}
\fig{irradiacionfg_sensibilidad}{figuras/fg/irradiacion_sensibilidad.pdf}
{Sensibilidad del FG polarizado con
    $V_{sd}$=\SI{100}{\milli\volt} en función de la dosis recibida.
}
La sensibilidad cambia con la dosis debido a dos fenómenos opuestos.
\begin{itemize}
    \item A medida que se descarga el floating gate, disminuye el 
        campo eléctrico en el óxido y baja el yield de generación de pares 
        electrón-hueco.
    \item Al reducirse la corriente, crece $\frac{dI_D}{dV_G}$ 
        (\figref{fig:didv}) y esto aumenta la sensibilidad.
\end{itemize}
Se ve en las mediciones un crecimiento de la sensibilidad que se aplana al
alzanzar \SI{50}{\Gray}.
\fig{didv}{figuras/fg/didv.pdf}
{La pendiente de la curva $I_D(V_G)$ del lector aumenta a medida que $I_D$ cae,
incrementando la sensibilidad del FG.
Esta curva fue simulada con modelos provistos por el foundry,
debido a que la compuerta del dispositivo fabricado está inaccesible para
mediciones.}
\subsection{Corriente de ruido}
Establecimos que una medición con este dosímetro consiste en promediar 10
muestras de corriente.
Así podemos definir el ruido como la desviación estándar de este promedio.
Medimos la corriente del lector en ausencia de radiación y
tomamos la diferencia entre muestras sucesivas para eliminar derivas
(\figref{fig:ruidofg}).
Esto resulta en una desviación estándar
$\sigma=$ \SI{27}{\nano\ampere},
correspondiente a una dosis de \SI{4}{\milli\gray}.
\fig{ruidofg}{figuras/fg/ruido.pdf}{Diferencia entre
mediciones sucesivas de corriente del lector, escaladas para representar el
ruido en promedios de 10 muestras.}
