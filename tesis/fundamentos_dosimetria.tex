\section{Dosimetry and radiation protection}
Radiation is an invisible part of many processes
and technologies which raise our standard of living:
nuclear energy production, medical diagnostics and therapy,
and scientific and industrial measurements.
Understanding its production, propagation and biological effects
is a starting point in order to protect people and the environment
from its associated risks.
\cite{iaea_radiation_????}.
This goal has led to the creation of organizations such as the
International Commision on Radiological Protection
\cite{_icrp_????} and the International Atomic Energy
Agency\cite{iaea_official_????}.
These entities compose recommendations for the safe use of radiation.
These are aimed at both operators and lawmakers.

In every context where radiation is used,
there are criteria which limit the exposure for
patients, workers and/or the general public.
These criteria are established as a balance between risks and benefits.
For example, radiotherapy can treat disease,
but it can also damage healthy tissue.
Additionaly, it is necessary to protect the operators and other personnel.

In order to limit exposure, 
there are practices which are aimed at different levels of an organization.
At the operator level,
manipulation of radioactive material is planned in order to minimize dose.
In addition, operator dose is monitored using various types of dosimeters,
and workplace contamination is monitored.
This is complemented by the creation of protocols for safe work
which include pre-established roles for emergency response.

At a higher level, each organization is responsible for creating a safety-oriented culture.
This implies worker involvement in the creation and implementation of safety norms,
while aiming for transparency and individual responsibility.

Above this, single organizations are regulated both externally
(by government or international bodies)
and internally (by peer inspection through entities such as the
World Association of Nuclear Operators
\cite{washington_practice_1997}).
\subsection{Dose}
The effects from radiation exposure vary with
\begin{itemize}
    \item particle fluence
        (number of particles crossing per unit time, per unit surface),
    \item exposure time,
    \item type of particle ($\alpha$, $\beta$, $\gamma$, etc.),
    \item substance of the target that is being irradiated,
    \item type of target structure (fatty tissue, DNA, pads or memory cells),
    \item target conditions (cell's stage in its lifecycle
        \cite{podgorsak_radiation_2005}, circuit biasing),
\end{itemize}etc.
As a starting point, in order to quantify the effects of ionizing radiation,
we define \emph{dose} as the energy deposited per unit mass.
It is typically measured in Gray 
(\SI{1}{\Gray} = \SI{1}{\joule\per\kilo\gram}) 
or rad
(\SI{1}{\rad} = \SI{0.01}{\Gray}).

In order to assess the effect on humans of one or more exposures,
we calculate an effective dose
\cite{martin_effective_2007}.
Each tissue has a weighing factor $w_t$ which corresponds to how severely it is affected by radiation.
Moreover, each type of particle has a weighing factor $w_r$
as a function of how damaging it is to a cell.
By weighing and summing each type of radiation and target tissue,
we calculate a number $E$ which more accurately reflects the total damage to the organism:
\begin{align*}
    E = \sum_r w_r \sum_t w_t D_{r,t}
\end{align*}
with $D_{r,t}$ the dose of particles $r$ entering tissue $t$.

Dose is an important measure due to its correlation with increased cancer probability.
The traditional model is called Linear Non-Threshold
\cite{valentin_low-dose_2006}.
It models the increase in cancer probability as a linear function of dose.
This is a simple model for radiation protection work.
However, it does not take into consideration any purported positive (hormetic)
effects at very low dose rates
\cite{hooker_linear_2004}.
