\section{Dosimetry and radiation protection}
Radiation is an invisible part of many processes
and technologies which raise our standard of living:
nuclear energy production, medical diagnosis and therapy,
and cientific and industrial measurements.
Understanding its production, 
La radiación es una parte invisible de muchos procesos 
y tecnologías que mejoran nuestra calidad de vida:
producción nuclear de energía, 
diagnóstico y terapias médicas, 
y mediciones con fines científicos e industriales.
Hay que entender su producción, propagación y efectos biológicos
como punto de partida para 
proteger a las personas y al ambiente de sus efectos deletéreos
\cite{iaea_radiation_????}.
Con este objetivo se crearon organizaciones como la International Commission on
Radiological Protection\cite{_icrp_????} y la International Atomic Energy
Agency\cite{iaea_official_????}.
Estos grupos crean recomendaciones para la seguridad en el uso de la radiación,
tanto para los operarios como para orientar la legislación en cada país.

En todo ámbito donde se emplea radiación 
hay criterios para la exposición máxima 
que pueden sufrir pacientes,
trabajadores y el público general.
Estos criterios se establecen balanceando los riesgos y beneficios.
Por ejemplo la radioterapia sirve para tratar la enfermedad
pero también puede dañar tejidos sanos.
Además es posible que haya riesgo para otras personas.
Por ejemplo, el personal que administra radiodiagnóstico y terapia.

Para limitar la exposición,
se definen prácticas a distintos niveles organizativos.
A nivel operativo, se planean las manipulaciones de material radioactivo
para minimizar la dosis.
Asimismo, se monitorea la dosis recibida por trabajadores 
mediante distintos tipos de dosímetros,
y se controla la contaminación del ambiente de trabajo.
Esto se acompaña con la creación de protocolos para el trabajo seguro
que incluyen roles pre-establecidos para responder a accidentes.

A nivel más alto, cada organización puede crear una cultura de seguridad:
involucrando a los trabajadores en la creación e implementación de las normas,
buscando transparencia y responsabilidad individual. 

Esto sigue con 
la regulación de las empresas, tanto de forma externa 
(gubernamental e internacional) como interna (revisión por pares mediante
organizaciones como la World Association of Nuclear Operators
\cite{washington_practice_1997}).
\subsection{Dosis}
Los efectos de la exposición a la radiación varían con 
\begin{itemize}
    \item flujo de partículas 
        (cantidad que cruzan por unidad de tiempo y superficie),
    \item tiempo de exposición,
    \item tipo de partícula incidente ($\alpha$, $\beta$, $\gamma$, etc.),
    \item sustancia donde incide,
    \item clase de estructuras presentes en el blanco (tejido graso o ADN,
        pads o celdas de memoria),
    \item condiciones del blanco (estadío de vida de una célula
        \cite{podgorsak_radiation_2005}, polarización de un circuito),
\end{itemize}etc.
Como punto de partida para cuantificar el efecto de la radiación ionizante
se define la \emph{dosis}: energía depositada por unidad de masa.
Unidades típicas son Gray 
(\SI{1}{\Gray} = \SI{1}{\joule\per\kilo\gram}) 
y rad 
(\SI{1}{\rad} = \SI{0.01}{\Gray}).

Para evaluar los efectos de una exposición o serie de exposiciones en humanos,
se calcula la dosis equivalente\cite{martin_effective_2007}.
Teniendo en cuenta las respuestas distintas de cada tejido,
se definen factores de peso $w_t$ para cada uno.
Asimismo, cada tipo de partícula tiene un factor de peso $w_r$
en función de su capacidad de dañar células.
Ponderando cada tipo y zona de radiación con estos factores se calcula un
número $E$ que representa de manera más precisa el daño total al organismo:
\begin{align*}
    E = \sum_r w_r \sum_t w_t D_{r,t}
\end{align*}
con $D_{r,t}$ la dosis de partículas $r$ recibida por el tejido $t$.

La dosis es una magnitud relevante debido a su correlación
con el aumento de probabilidad de cáncer.
El modelo usado convencionalmente se llama Linear Non-Threshold
\cite{valentin_low-dose_2006}.
Supone una relación lineal entre dosis y el aumento de probabilidad de cáncer.
Esta es una base simple para el trabajo en protección.
Sin embargo no modela posibles efectos positivos (horméticos) 
de dosis muy bajas \cite{hooker_linear_2004}.
