\section{Radiación}
\label{sec:radiacion}
La radiación es el transporte de energía mediado por distintas partículas.
En nuestro caso tratamos con fotones (rayos X) y electrones (rayos $\beta$).
Por encima de cierto umbral de energía, 
las interacciones de estas partículas con la materia
son capaces de ionizarla.
Esto produce daños tanto en tejidos orgánicos como en circuitos electrónicos.
%
\subsection{Radiación $\alpha$}
Las partículas $\alpha$ son núcleos de $^4$He.
Son producidas por núcleos inestables como
$^{241}$Am (usado en detectores de humo) y U 
cuando sufren decaimiento $\alpha$.
Este proceso genera partículas con energías cercanas a 
\SI{5}{\mega\electronvolt},
que se detienen en algunos centímetros de aire 
o en la capa de células muertas de la epidermis.
No son una fuente significativa de dosis a humanos,
salvo en caso de inhalación o ingestión,
o de tener energías altas
(por ejemplo si provienen de rayos cósmicos).
\subsection{Neutrones}
Los neutrones son partículas neutras presentes en los núcleos atómicos.
Los distintos isótopos de cada elemento corresponden a núcleos que difieren
sólo en la cantidad de neutrones.
Los procesos de fusión y fisión nuclear 
tienden a emitir neutrones energéticos o,
en procesos de nucleosíntesis,
capturar neutrones para formar núcleos más masivos.

A fines de protección, se estudian 3 tipos de interacciones entre neutrones
y los núcleos de un blanco:
scattering elástico e inelástico, y captura.
En el scattering inelástico,
parte de la energía inicial excita el núcleo del blanco y es luego emitida como
fotones.
La captura se da en los neutrones de menor energía,
frenados previamente por scattering,
y está acompañada por emisión de rayos $\gamma$.
Por ejemplo, en la terapia por captura neutrónica con Boro,
se hace llegar boro a células cancerígenas para que capture neutrones.
Así los tumores son destruídos por los fotones $\gamma$ y las partículas 
$\alpha$ resultantes del proceso de captura.
\subsection{Radiación $\beta$}
La radiación $\beta$ consiste en electrones energéticos.
Los mismos pasan por la materia y depositan energía por procesos de ionización 
o la emiten como radiación de frenado (bremsstrahlung).
Esto limita su rango o profundidad a la que pueden penetrar un material antes
de agotar su energía.
En aire recorren distancias típicas de metros, 
mientras que en materiales más densos recorren milímetros.

La materia con la que interactuamos tiene una gran fracción de electrones 
(en número).
Sin embargo, estos electrones se encuentran ligados a protones y neutrones,
incapaces de escapar por falta de energía.
Cuando incide radiación $\beta$ o $\gamma$ sobre un material,
puede producir electrones libres (radiación $\beta$ secundaria)
suponiendo que la partícula incidente supera la energía de ionización del
material.

Otra forma de producir radiación $\beta$ es crear nuevos electrones a partir de
energía.
Esto se da naturalmente en algunos isótopos radioactivos en un proceso llamado
decaimiento $\beta$.
\subsubsection{Decaimiento $\beta$}
El decaimiento $\beta$ es un proceso
que convierte un núcleo $X$ en otro menos masivo $X'$ y emite
un antineutrino electrónico y un electrón.
Por ejemplo, el decaimiento del estroncio a itrio
\begin{align*}
    \left.^{90}\strontium\right.
    &\to\left.^{90}\yttrium\right.+e^-+\overline\nu_e.
\end{align*}
La variación de la cantidad de átomos $N_X$ puede modelarse con
\begin{align*}
    \deriv{N_X}t &= -\lambda N_X\\
    \deriv{N_{X'}}t &= \lambda N_X
\end{align*}
si $X'$ es estable. 
$\lambda$ es un parámetro que determina la tasa de decaimiento.
La solución es
\begin{align}
    \label{eq:soluciondecaimiento}
    N_X(t) &= N_X(0)e^{-\lambda t}\\
    N_{X'}(t) &= N_X(0)(1-e^{-\lambda t}).\nonumber
\end{align}
En vez de $\lambda$ se habla comunmente de la vida media de $X$:
el tiempo $\tau$ que tarda $N_X$ en caer a la mitad.
Puede despejarse de la \equref{eq:soluciondecaimiento}
\begin{align*}
    \frac 1 2 N_X(0) &= N_X(0)e^{-\lambda\tau}\\
    \tau &= \frac{\ln2}\lambda
\end{align*}
\subsubsection{Equilibrio secular}
El nuevo núcleo creado puede ser inestable.
Por ejemplo, el $^{90}\strontium$ decae en $^{90}\yttrium$ que a su vez decae
en $^{90}\zirconium$.
Modelamos esto con
\begin{align*}
    \deriv{N_\strontium}t &= -\lambda_\strontium N_\strontium\\
    \deriv{N_\yttrium}t &= \lambda_\strontium N_\strontium
        -\lambda_\yttrium N_\yttrium\\
    \deriv{N_\zirconium}t &= \lambda_\yttrium N_\yttrium.
\end{align*}
Su solución es
\begin{align*}
    N_\strontium(t) &= N_\strontium(0)e^{-\lambda_\strontium t}\\
    %\deriv{N_\yttrium}t +\lambda_\yttrium N_\yttrium &= 
        %\lambda_\strontium N_\strontium(0)e^{-\lambda_\strontium t}
    N_\yttrium(t) &= ae^{-\lambda_\yttrium t}
        +N_\strontium(0)\frac{\lambda_\strontium}
        {\lambda_\yttrium-\lambda_\strontium}e^{-\lambda_\strontium t}.
\end{align*}
Dado que la vida media del $^{90}\yttrium$ (2.7 días) 
es mucho menor que la del $^{90}\strontium$ (29 años),
tenemos $\lambda_\yttrium\gg\lambda_\strontium$.
Las cantidades de cada elemento tienden a un equilibrio secular dado por
\begin{align*}
    N_\strontium(t) &= N_\strontium(0)e^{-\lambda_\strontium t}\\
    N_\yttrium(t) &\approx N_\strontium(0)
        \frac{\lambda_\strontium}{\lambda_\yttrium}
        e^{-\lambda_\strontium t}.
\end{align*}La frecuencia con que cada elemento emite un electrón es su
actividad. 
Se expresa en Becquerel
(\SI{1}{\becquerel}=\SI{1}{\per\second}) o Curie
(1 Ci=\SI{3.7e10}{\per\second}), y está dada en este ejemplo por
\begin{align*}
    A_\strontium(t) &= N_\strontium(0)\lambda_\strontium
        e^{-\lambda_\strontium t}\\
    A_\yttrium(t) &\approx N_\strontium(0)
        \frac{\lambda_\strontium}{\lambda_\yttrium}\lambda_\yttrium
        e^{-\lambda_\strontium t}=A_\strontium(t).
\end{align*}
Por lo tanto, una fuente de $^{90}\strontium$ en equilibrio secular
debe mitad de su actividad a decaimientos $\strontium\to\yttrium$
y mitad a $\yttrium\to\zirconium$.
%
\subsubsection{Frenado de electrones}
La pérdida de energía de los electrones se debe a interacciones discretas con
los átomos de un escudo.
Aproximando el frenado como una variación contínua de su energía 
en función de la distancia $E(x)$,
podemos plantear un poder de frenado $S(E)$ del material tal que
\begin{align*}
    \frac{dE}{dx}=-S(E).
\end{align*}
Se llama poder de frenado de masa a $S/\rho$. 
Esta es una medida de uso común en protección de radiación,
que permite calcular tasa de dosis de un haz usando
\begin{align*}
    \dot D&=\deriv{E}{mdt}=\frac 1\rho\deriv{E}{x}\deriv{N}{Adt}=\frac{IS}\rho
\end{align*}
con $I$ el flujo de partículas por unidad de área (intensidad)
y $\dot D$ la deposición de energía por unidad de masa y de tiempo 
(tasa de dosis).
$S$ está tabulada para cada elemento (\figref{fig:stopping})
\fig{stopping}{figuras/rad/stopping.pdf}{Poder de frenado de electrones 
para distintos materiales en función de la energía \cite{berger_estar_????}.}
junto con el rango o distancia de frenado
\begin{align*}
    R(E) &= \int_0^E \frac{dE'}{S(E')}
\end{align*} y la eficiencia radiativa $Y$.
Esta es la fracción de la energía electrónica que se 
emite como fotones (\emph{bremsstrahlung} o radiación de frenado). 

$Y$ aumenta con el peso atómico del material de frenado.
Para fines de protección,
se suelen frenar los electrones con polímeros como acrílico que tienen bajo
rendimiento radiativo.
Así se minimiza la producción de rayos X y se reduce el espesor de plomo
necesario para atenuarlos.
%
\subsection{Radiación X}
Los rayos X son fotones de longitud de onda menor a \SI{10}{\nano\meter}.
Transportan energía suficiente para ionizar muchos materiales.
Se producen típicamente por el frenado de partículas cargadas al interactuar
con la materia.
\subsubsection{Radiación de frenado (bremsstrahlung)}
Cuando una carga es acelerada, emite radiación
electromagnética\cite{jackson_classical_1998}.
Un caso importante es la bremsstrahlung,
radiación por frenado de electrones en materia.
Ocurre en la naturaleza por interacción de rayos cósmicos con la atmósfera,
y artificialmente en los tubos de rayos X 
al frenar electrones en un blanco metálico.

La densidad espectral de potencia de bremsstrahlung está dada por la ley de
Kramers\cite{kramers_xciii._1923}
\begin{align}
    P(E)dE = \frac{2P_T}{E_M^2}(E_M-E)dE
    \label{eq:kramers}
\end{align} para electrones con energía $E_M$ (\figref{fig:kramers}).
\fig{kramers}{figuras/rad/kramers.pdf}
{Densidad espectral de potencia (PSD) de la radiación proveniente del 
frenado de electrones con energía $E_M$ irradiando una potencia total $P_T$.}
$P_T$ es la potencia total de \emph{bremsstrahlung}, dada por
\begin{align*}
    P_T&=E_MIY(E_M)
\end{align*}con $I$ la intensidad de electrones e $Y$ la eficiencia radiativa a
esa energía.
\subsubsection{Absorción de X}
Modelamos la intensidad de rayos X como función contínua de la energía y
posición $I(E,x)$.
Así podemos plantear un coeficiente de absorción $\mu(E)$ para cada 
material (\figref{fig:absorcionX}) tal que 
\fig{absorcionX}{figuras/rad/absorcionX.pdf}
{Tasa de absorción $\mu$ de rayos X en función de la energía\cite{xraycoef}.}
\begin{align}
    \label{eq:absorcionx}
    dI/dx=-\mu I.
\end{align}
Este $\mu$ se encuentra tabulado para distintos elementos y materiales
\cite{xraycoef}.
\subsubsection{Factor de buildup}
% DETAIL: de repente estoy hablando de escudos y detectores
El $\mu$ considera tanto absorción de fotones como scattering en un cuerpo.
Si el detector (o la zona donde estoy calculando dosis) 
está cerca de la fuente, 
algunos fotones dispersados pueden impactarlo (\figref{fig:buildup}).
\fig{buildup}{figuras/buildup/buildup.pdf}{Geometría mala: cerca del escudo algunos fotones
dispersados llegan al detector.}
Por eso se tabula el factor de buildup\cita{martin_physics_2013}
que relaciona la verdadera intensidad a la salida del escudo 
con la calculada mediante la \equref{eq:absorcionx}.
Su valor se obtiene numéricamente mediante simulaciones Monte-Carlo
(capítulo~\ref{montecarlo})
que tienen en cuenta tanto absorción como scattering
para una geometría dada.
