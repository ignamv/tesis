\section{Radiation}
\label{sec:radiacion}
Radiation is the transport of energy, mediated by different particles.
This thesis deals with photons ($\gamma$ rays) and electrons ($\beta$ rays).
When these particles carry enough energy,
they interact with matter in such a way to cause ionization.
This can lead to damage both in living tissue and in electronic devices.
%
\subsection{$\alpha$ radiation}
$\alpha$ particles are $^4$He nuclei.
They are produced by unstable nuclei such as
$^{241}$Am (which is used in smoke detectors) and U,
in a process known as $\alpha$ decay.
This process can generate particles with energies close to
\SI{5}{\mega\electronvolt},
which are stopped after traveling through a few centimeters of air
or the layer of dead cells of the epidermis.
They are typically not hazardous for humans,
except when inhaled or ingested,
or if they have very high energies (eg. from cosmic rays)
\subsection{Neutrons}
Neutrons are neutral particles which are present in atomic nuclei.
The various isotopes of each element are nuclei which only differ in the number of neutrons.
Nuclear fusion and fission usually emit high-energy neutrons or,
in nucleosynthesis processes,
capture neutrons and thus form heavier nuclei.

For protection purposes, the relevant interactions between neutrons and nuclei
are elastic scattering, inelastic scattering and capture.
In inelastic scattering,
part of the incoming neutron's energy excites the target nuclei to a higher energy level.
This energy is later emitted as photons when the nucleus relaxes.
Capture happens with lower energy neutrons,
for example those which have lost energy to scattering,
and produces $\gamma$ rays
For example, in boron neutron capture therapy,
boron is delivered to cancer cells.
When irradiated with neutrons, these cells are destroyed
by the $\alpha$ and $\gamma$ radiation produced by the capture process.
\subsection{$\beta$ radiation}
$\beta$ consists of high-energy electrons.
These can travel through matter and deposit energy through ionization,
or emit energy as braking radiation (bremsstrahlung)
This limits the range or penetration depth into a material before they run out of energy.
Their range in air is in the order of meters,
while denser materials can restrict their range to millimeters.

Everyday matter is largely composed of electrons (by number).
However, these electrons are bound to atomic nuclei,
and lack the energy to escape the potential well.
When $\beta$ or $\gamma$ radiation strikes a material,
it can produce free electrons (secondary $\beta$ radiation)
if the incoming particle energy exceeds the material ionization energy.

Otra forma de producir radiación $\beta$ es crear nuevos electrones a partir de
energía.
Esto se da naturalmente en algunos isótopos radioactivos en un proceso llamado
decaimiento $\beta$.
\subsubsection{Decaimiento $\beta$}
El decaimiento $\beta$ es un proceso
que convierte un núcleo $X$ en otro menos masivo $X'$ y emite
un antineutrino electrónico y un electrón.
Por ejemplo, el decaimiento del estroncio a itrio
\begin{align*}
    \left.^{90}\strontium\right.
    &\to\left.^{90}\yttrium\right.+e^-+\overline\nu_e.
\end{align*}
La variación de la cantidad de átomos $N_X$ puede modelarse con
\begin{align*}
    \deriv{N_X}t &= -\lambda N_X\\
    \deriv{N_{X'}}t &= \lambda N_X
\end{align*}
si $X'$ es estable. 
$\lambda$ es un parámetro que determina la tasa de decaimiento.
La solución es
\begin{align}
    \label{eq:soluciondecaimiento}
    N_X(t) &= N_X(0)e^{-\lambda t}\\
    N_{X'}(t) &= N_X(0)(1-e^{-\lambda t}).\nonumber
\end{align}
En vez de $\lambda$ se habla comunmente de la vida media de $X$:
el tiempo $\tau$ que tarda $N_X$ en caer a la mitad.
Puede despejarse de la \equref{eq:soluciondecaimiento}
\begin{align*}
    \frac 1 2 N_X(0) &= N_X(0)e^{-\lambda\tau}\\
    \tau &= \frac{\ln2}\lambda
\end{align*}
\subsubsection{Equilibrio secular}
El nuevo núcleo creado por decaimiento $\beta$ puede, a su vez, ser inestable.
Por ejemplo, el $^{90}\strontium$ decae en $^{90}\yttrium$ que decae
en $^{90}\zirconium$.
Modelamos esto con
\begin{align*}
    \deriv{N_\strontium}t &= -\lambda_\strontium N_\strontium\\
    \deriv{N_\yttrium}t &= \lambda_\strontium N_\strontium
        -\lambda_\yttrium N_\yttrium\\
    \deriv{N_\zirconium}t &= \lambda_\yttrium N_\yttrium.
\end{align*}
Su solución es
\begin{align*}
    N_\strontium(t) &= N_\strontium(0)e^{-\lambda_\strontium t}\\
    %\deriv{N_\yttrium}t +\lambda_\yttrium N_\yttrium &= 
        %\lambda_\strontium N_\strontium(0)e^{-\lambda_\strontium t}
    N_\yttrium(t) &= ae^{-\lambda_\yttrium t}
        +N_\strontium(0)\frac{\lambda_\strontium}
        {\lambda_\yttrium-\lambda_\strontium}e^{-\lambda_\strontium t}.
\end{align*}
Dado que la vida media del $^{90}\yttrium$ (2.7 días) 
es mucho menor que la del $^{90}\strontium$ (29 años),
tenemos $\lambda_\yttrium\gg\lambda_\strontium$.
Las cantidades de cada elemento tienden a un equilibrio secular dado por
\begin{align*}
    N_\strontium(t) &= N_\strontium(0)e^{-\lambda_\strontium t}\\
    N_\yttrium(t) &\approx N_\strontium(0)
        \frac{\lambda_\strontium}{\lambda_\yttrium}
        e^{-\lambda_\strontium t}.
\end{align*}La frecuencia con que cada elemento emite un electrón es su
actividad. 
Se expresa en Becquerel
(\SI{1}{\becquerel}=\SI{1}{\per\second}) o Curie
(1 Ci=\SI{3.7e10}{\per\second}), y está dada en este ejemplo por
\begin{align*}
    A_\strontium(t) &= N_\strontium(0)\lambda_\strontium
        e^{-\lambda_\strontium t}\\
    A_\yttrium(t) &\approx N_\strontium(0)
        \frac{\lambda_\strontium}{\lambda_\yttrium}\lambda_\yttrium
        e^{-\lambda_\strontium t}=A_\strontium(t).
\end{align*}
Por lo tanto, una fuente de $^{90}\strontium$ en equilibrio secular
debe mitad de su actividad a decaimientos $\strontium\to\yttrium$
y mitad a $\yttrium\to\zirconium$.
%
\subsubsection{Frenado de electrones}
La pérdida de energía de los electrones se debe a interacciones discretas con
los átomos de un escudo.
Aproximando el frenado como una variación contínua de su energía 
en función de la distancia $\epsilon(x)$,
podemos plantear un poder de frenado $S(\epsilon)$ del material tal que
\begin{align*}
    \frac{d\epsilon}{dx}=-S(\epsilon).
\end{align*}
con $\frac{d\epsilon}{dx}$ la variación de energía de un electrón por unidad de
distancia.

En protección de radiación no se suele trabajar con electrones individuales
sino con haces. Típicamente se considera un haz con un flujo constante de
$\frac{dN}{dt}$ partículas por unidad de tiempo.
También se define su intensidad $I=\frac{dN}{dAdt}$,
que es el flujo de partículas por unidad de área.

Para calcular la tasa de dosis de un haz,
resulta útil definir el poder de frenado de masa $S/\rho$. 
Así se llega a una expresión simple para la tasa de dosis:
\begin{align*}
    \dot D&=\deriv{}{t}\deriv{E}{m}=\deriv{\epsilon dN}{mdt}=\frac
    1\rho\deriv{\epsilon}{x}\deriv{N}{Adt}=
    \frac{IS}\rho
\end{align*}
con $\rho$ la densidad del material bajo irradiación
y $\dot D$ la deposición de energía por unidad de masa y de tiempo 
(tasa de dosis).
$S$ está tabulada para cada elemento y material (\figref{fig:stopping})
\fig{stopping}{figuras/rad/stopping.pdf}{Poder de frenado de electrones 
para distintos materiales en función de la energía.
Valores publicados por NIST en el proyecto ESTAR\cite{berger_estar_????}.}
junto con el rango o distancia de frenado
\begin{align*}
    R(\epsilon) &= \int_0^\epsilon \frac{d\epsilon'}{S(\epsilon')}
\end{align*} y la eficiencia radiativa $Y$.
Esta es la fracción de la energía electrónica que se 
emite como fotones (\emph{bremsstrahlung} o radiación de frenado). 

$Y$ aumenta con el peso atómico del material de frenado.
Para fines de protección,
se suelen frenar los electrones con polímeros como acrílico que tienen bajo
rendimiento radiativo.
Así se minimiza la producción de rayos X y se reduce el espesor de plomo
necesario para atenuarlos.
%
\subsection{Radiación X}
Los rayos X son fotones de longitud de onda menor a \SI{10}{\nano\meter}.
Transportan energía suficiente para ionizar muchos materiales.
Se producen típicamente por el frenado de partículas cargadas al interactuar
con la materia.
\subsubsection{Radiación de frenado (bremsstrahlung)}
Cuando una carga es acelerada, emite radiación
electromagnética\cite{jackson_classical_1998}.
Un caso importante es la bremsstrahlung,
radiación por frenado de electrones en materia.
Ocurre en la naturaleza por interacción de rayos cósmicos con la atmósfera,
y artificialmente en los tubos de rayos X 
al frenar electrones en un blanco metálico.

La densidad espectral de potencia de bremsstrahlung está dada por la ley de
Kramers\cite{kramers_xciii._1923}
\begin{align}
    P(E)dE = \frac{2P_T}{E_M^2}(E_M-E)dE
    \label{eq:kramers}
\end{align} para electrones con energía $E_M$ (\figref{fig:kramers}).
\fig{kramers}{figuras/rad/kramers.pdf}
{Densidad espectral de potencia (PSD) de la radiación proveniente del 
frenado de electrones con energía $E_M$ irradiando una potencia total $P_T$.}
$P_T$ es la potencia total de \emph{bremsstrahlung}, dada por
\begin{align*}
    P_T&=E_MIY(E_M)
\end{align*}con $I$ la intensidad de electrones e $Y$ la eficiencia radiativa a
esa energía.
\subsubsection{Absorción de X}
Modelamos la intensidad de rayos X como función contínua de la energía y
posición $I(E,x)$.
Así podemos plantear un coeficiente de absorción $\mu(E)$ para cada 
material (\figref{fig:absorcionX}) tal que 
\fig{absorcionX}{figuras/rad/absorcionX.pdf}
{Tasa de absorción $\mu$ de rayos X en función de la energía.
    Valores tabulados por NIST\cite{xraycoef}.}
\begin{align}
    \label{eq:absorcionx}
    dI/dx=-\mu I.
\end{align}
Este $\mu$ se encuentra tabulado para distintos elementos y materiales
\cite{xraycoef}.
\subsubsection{Factor de buildup}
% DETAIL: de repente estoy hablando de escudos y detectores
El $\mu$ considera tanto absorción de fotones como scattering en un cuerpo.
Si el detector (o la zona donde estoy calculando dosis) 
está cerca de la fuente, 
algunos fotones dispersados pueden impactarlo (\figref{fig:buildup}).
\fig{buildup}{figuras/buildup/buildup.pdf}{Geometría mala: cerca del escudo algunos fotones
dispersados llegan al detector.}
Por eso se tabula el factor de buildup\cita{martin_physics_2013}
que relaciona la verdadera intensidad a la salida del escudo 
con la calculada mediante la \equref{eq:absorcionx}.
Su valor se obtiene numéricamente mediante simulaciones Monte-Carlo
(capítulo~\ref{montecarlo})
que tienen en cuenta tanto absorción como scattering
para una geometría dada.
