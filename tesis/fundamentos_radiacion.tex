\section{Radiation}
\label{sec:radiacion}
Radiation is the transport of energy, mediated by different particles.
This thesis deals mainly with photons (X rays) and electrons ($\beta$ rays).
When these particles carry enough energy,
they interact with matter in such a way to cause ionization.
This can lead to damage both in living tissue and in electronic devices.
%
\subsection{$\alpha$ radiation}
$\alpha$ particles are $^4$He nuclei.
They are produced by unstable nuclei such as
$^{241}$Am (which is used in smoke detectors) and U,
in a process known as $\alpha$ decay.
This process can generate particles with energies close to
\SI{5}{\mega\electronvolt},
which are stopped after traveling through a few centimeters of air
or the layer of dead cells on the epidermis.
They are typically not hazardous for humans,
except when inhaled or ingested,
or if they have very high energies (eg. from cosmic rays)
\subsection{Neutrons}
Neutrons are neutral particles which are present in atomic nuclei.
The various isotopes of each element are nuclei which only differ in the number of neutrons.
Nuclear fusion and fission usually emit high-energy neutrons or,
in nucleosynthesis processes,
capture neutrons and thus form heavier nuclei.

For protection purposes, the relevant interactions between neutrons and nuclei
are elastic scattering, inelastic scattering and capture.
In inelastic scattering,
part of the incoming neutron's energy excites the target nuclei to a higher energy level.
This energy is later emitted as photons when the nucleus relaxes.
Capture happens with lower energy neutrons,
for example those which have lost energy to scattering,
and produces $\gamma$ rays
For example, in boron neutron capture therapy,
boron is delivered to cancer cells.
When irradiated with neutrons, these cells are destroyed
by the $\alpha$ and $\gamma$ radiation produced by the capture process.
\subsection{$\beta$ radiation}
$\beta$ radiation consists of high-energy electrons.
These can travel through matter and deposit energy through ionization,
or emit energy as braking radiation (bremsstrahlung)
This limits the range or penetration depth into a material before they run out of energy.
Their range in air is in the order of meters,
while denser materials can restrict their range to millimeters.

Everyday matter is largely composed of electrons (by number).
However, these electrons are bound to atomic nuclei,
and lack the energy to escape the potential well.
When $\beta$ or $\gamma$ radiation strikes a material,
it can produce free electrons (secondary $\beta$ radiation)
if the incoming particle energy exceeds the material ionization energy.

$\beta$ radiation can also be produced by decaying radioactive isotopes,
in a process called $\beta$ decay.
\subsubsection{$\beta$ decay}
$\beta$ decay is a process whereby an atomic nucleus $X$ 
turns into a lighter nucleus $X'$,
emitting an electron and an electron antineutrino.
For example, strontium decaying to Yttrium:
\begin{align*}
    \left.^{90}\strontium\right.
    &\to\left.^{90}\yttrium\right.+e^-+\overline\nu_e.
\end{align*}
The variation in the number of atoms $N_X$ as a function of time follows the equation
\begin{align*}
    \deriv{N_X}t &= -\lambda N_X\\
    \deriv{N_{X'}}t &= \lambda N_X
\end{align*}
if $X'$ is stable over the timescales of interest.
$\lambda$ is a parameter which sets the decay rate.
In the context of radiation, this decay rate is called activity
and is measured in Becquerel (\SI{1}{\becquerel}=\SI{1}{\per\second})
or Curie (1 Ci=\SI{3.7e10}{\per\second}).
The solutions to this equation are of the form
\begin{align}
    \label{eq:soluciondecaimiento}
    N_X(t) &= N_X(0)e^{-\lambda t}\\
    N_{X'}(t) &= N_X(0)(1-e^{-\lambda t}).\nonumber
\end{align}
Rather than $\lambda$, it is more usual to speak of the half life of $X$.
This is the time $\tau$ it takes for $N_X$ to halve.
Its relation to $\lambda$ can be deduced from \equref{eq:soluciondecaimiento}.
\begin{align*}
    \frac 1 2 N_X(0) &= N_X(0)e^{-\lambda\tau}\\
    \tau &= \frac{\ln2}\lambda
\end{align*}
\subsubsection{Secular equilibrium}
The lighter nucleus produced by $\beta$ decay may itself be unstable.
For example, $^{90}\strontium$ decays to $^{90}\yttrium$ 
which in turn decays to $^{90}\zirconium$.
The quantities of each element follow the equations
\begin{align*}
    \deriv{N_\strontium}t &= -\lambda_\strontium N_\strontium\\
    \deriv{N_\yttrium}t &= \lambda_\strontium N_\strontium
        -\lambda_\yttrium N_\yttrium\\
    \deriv{N_\zirconium}t &= \lambda_\yttrium N_\yttrium.
\end{align*}
The solutions are of the form
\begin{align*}
    N_\strontium(t) &= N_\strontium(0)e^{-\lambda_\strontium t}\\
    %\deriv{N_\yttrium}t +\lambda_\yttrium N_\yttrium &= 
        %\lambda_\strontium N_\strontium(0)e^{-\lambda_\strontium t}
    N_\yttrium(t) &= ae^{-\lambda_\yttrium t}
        +N_\strontium(0)\frac{\lambda_\strontium}
        {\lambda_\yttrium-\lambda_\strontium}e^{-\lambda_\strontium t}.
\end{align*}
Because $^{90}\yttrium$'s half life (2.7 days) 
is much smaller than $^{90}\strontium$'s (29 years),
it holds that $\lambda_\yttrium\gg\lambda_\strontium$.
Therefore, the quantities of both elements reach a secular equilibrium given by
\begin{align*}
    N_\strontium(t) &= N_\strontium(0)e^{-\lambda_\strontium t}\\
    N_\yttrium(t) &\approx N_\strontium(0)
        \frac{\lambda_\strontium}{\lambda_\yttrium}
        e^{-\lambda_\strontium t}.
\end{align*}

The activity is
\begin{align*}
    A_\strontium(t) &= N_\strontium(0)\lambda_\strontium
        e^{-\lambda_\strontium t}\\
    A_\yttrium(t) &\approx N_\strontium(0)
        \frac{\lambda_\strontium}{\lambda_\yttrium}\lambda_\yttrium
        e^{-\lambda_\strontium t}=A_\strontium(t).
\end{align*}
Therefore, a $^{90}\strontium$ source in secular equilibrium
owes half of its activity to $\strontium\to\yttrium$ decays
and half to $\yttrium\to\zirconium$ decays.
%
\subsubsection{Electronic stopping}
Electrons lose energy through discrete interactions with the atoms that make up matter.
This series of interactions can be approximated as
a continuous variation of energy with distance $\epsilon(x)$.
This allows us to define a stopping power $S(\epsilon)$ for each material such that
\begin{align*}
    \frac{d\epsilon}{dx}=-S(\epsilon).
\end{align*}

One is typically interested in studying protection from a beam of electrons,
rather than a single electron.
Beams can be characterized by their particle flux (number of particles per unit time)
$\dot N=\frac{dN}{dt}$
or their fluence rate (number of particles per unit time, per unit area)
$\dot\Phi=\frac{dN}{dAdt}$.

In calculating a beam's dose rate (deposited energy per unit time, per unit mass),
one typically works with the mass stopping power $S/\rho$,
which is the stopping power divided by the material density.
This leads to a simple expression for the dose rate:
\begin{align*}
    \dot D&=\deriv{}{t}\deriv{E}{m}=\deriv{\epsilon dN}{mdt}=\frac
    1\rho\deriv{\epsilon}{x}\deriv{N}{Adt}=
    I\frac{S}\rho
\end{align*}
$S$ and $S/\rho$ are tabulated for many elements and materials (\figref{fig:stopping}).
\fig{stopping}{figuras/rad/stopping.pdf}{Electronic stopping power
for different materials as a function of energy.
Values published by NIST's ESTAR project\cite{berger_estar_????}.}
Another tabulated quantity is the range,
which is how far they can travel through a material in a straight line
before exhausting their energy:
\begin{align*}
    R(\epsilon) &= \int_0^\epsilon \frac{d\epsilon'}{S(\epsilon')}
\end{align*}

Finally, there is the radiation yield $Y$.
This is the fraction of energy which is emitted as photons
(braking radiation or bremsstrahlung).
$Y$ is an increasing function of atomic number.
For protection purposes, it is common to stop electrons with
polymers such as acrylic, which have low radiation yield.
This minimizes X ray production,
meaning less lead is required to absorb the braking radiation.
%
\subsection{X radiation}
X rays are photons with wavelengths between \SI{0.03}{\nano\meter} and \SI{10}{\nano\meter}.
They carry sufficient energy to ionize many materials.
They are typically produced when charged particles interact with matter.
\subsubsection{Braking radiation (bremsstrahlung)}
When a charged particle accelerates, it emits electromagnetic radiation
\cite{jackson_classical_1998}.
Bremsstrahlung is an example of this,
produced by electrons which slow down in matter.
It occurs naturally as cosmic rays interact with the atmosphere,
and artificially in X ray tubes as electrons impact a metallic target.

Bremsstrahlung photons have a power spectral density given by
Kramers' law\cite{kramers_xciii._1923}
\begin{align}
    P(E)dE = \frac{2P_T}{E_M^2}(E_M-E)dE
    \label{eq:kramers}
\end{align} with $E_M$ the incoming electron energy (\figref{fig:kramers}).
\fig{kramers}{figuras/rad/kramers.pdf}
{Power spectral density produced by stopping electrons with energy $E_M$,
with total power $P_T$.}
$P_T$ is the total bremsstrahlung power, given by
\begin{align*}
    P_T&=E_MIY(E_M)
\end{align*} with $I$ the electron flux
and $Y$ the radiation yield for the target material at that energy.
\subsubsection{X ray shielding}
We model X ray intensity as a continuous function of energy and position $I(E,x)$.
This allows us to define an absorption coefficient $\mu(E)$ for each material
(\figref{fig:absorcionX}) such that
\fig{absorcionX}{figuras/rad/absorcionX.pdf}
{X ray absorption coefficient $\mu$ as a function of energy.
    Values tabulated by NIST\cite{xraycoef}.}
\begin{align}
    \label{eq:absorcionx}
    dI/dx=-\mu I.
\end{align}
$\mu$ is tabulated for many elements and materials\cite{xraycoef}.
\subsubsection{Buildup factor}
% DETAIL: de repente estoy hablando de escudos y detectores
The absorption coefficient $\mu$ takes into account both photon absorption and scattering in a body.
When one calculates the dose rate close to the shield,
some of the photons scattered by the shield may still reach the region of interest
(\figref{fig:buildup}).
\fig{buildup}{figuras/buildup/buildup.pdf}{Counterproductive geometry:
some of the photons scattered by the shield still reach the detector.}
For this reason, there are tabulated values for the buildup factor\cita{martin_physics_2013},
which is the ratio of the actual intensity in the detector
to the intensity calculated using \equref{eq:absorcionx}.
The buildup factor is found numerically, using Monte Carlo simulations
(chapter~\ref{montecarlo}),
which take into account both scattering and absorption
for a given geometry.
