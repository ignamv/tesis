\subsection{Geant4}
Geant4 is a package of open-source tools
to simulate interactions between particles and matter.
It is developed as a collaboration between many institutes
such as CERN, ESA and Fermilab.
It is used in many subfields of physics, medicine and space engineering.
\subsubsection{Program structure}
Geant4 is a C++ library.
By implementing interfaces defined in this library,
one can define classes which specify:
\begin{itemize}
    \item Particle sources: the distribution in location, angle, energy and particle type
    \item Detector and environment geometry and materials
    \item Interaction logging: what magnitudes or statistics
        to store to disk or filter out. For example, one might want to exclude certain kinds of particles.
\end{itemize}
These classes can then be instanced and passed as arguments to Geant4 functions,
which run the simulation.

It is possible for these user-defined classes to read settings from a plain-text command file.
This allows for fast experimentation, running variations on a simulation
without the need to recompile the program (only changing the command file).
\subsubsection{Geometry}
In order to define the problem geometry,
I used the 3D models I created in FreeCAD when building the irradiator.
Geant4 allows the user to define the simulation environment with a GDML file\cite{chytracek_geometry_2006}
using the \code{G4GDMLParser} class.
I wrote a Python script to convert 
FreeCAD \code{.fcstd} files to GDML.
It allows the user to define the material for each solid
(using the material's name in the Geant4 database)
and which volume makes up the detector.

To simplify usage, I created the class 
\code{GDMLDetectorConstruction},
which can be configured from the command file.
For example, the following lines load a GDML file
and set the sensitive volume by name:
\begin{verbatim}
/gdml/load pvc_pb.gdml
/gdml/sensitivevolume detector_volume_name
\end{verbatim}

\subsubsection{Particle source}
Geant4 provides the class \code{G4GeneralParticleSource},
which can be configured for different types of sources:
point or extended, which various beam shapes and energy spectra.
These settings are read from the command file at runtime.
For example, the following lines create a disk-shaped source
shooting particles in the z direction:
\begin{verbatim}
/gps/pos/type Plane
/gps/pos/shape Circle
/gps/pos/radius 3 mm
/gps/ang/type beam1d
/gps/direction 0 0 1
\end{verbatim}
\subsubsection{Interaction logging}
Each primary particle generated by the source
causes interactions all across the problem geometry.
However, we are normally only interested in those interactions which
take place inside a bounded volume (detector).
For example, in radiation protection,
we are interested in how much dose reaches the human body.
In dosimetry, we might be interested in how much charge is generated
within the depletion region of a junction.
All other interactions required to simulate the path of radiation 
between the source and the detector are not of interest,
and would take up excessive storage space if logged.

The straightforward solution is to log the total energy deposited in the detector.
This is the minimum amount of information required to calculate dose.
I chose to log the coordinates and deposited energy of all interactions within the detector.
This does not take up excessive disk space,
and allows for more flexibility when post-processing the results.
For example, it allows for an analysis of how dose varies with depth.
Interactions are first written to memory using class \code{G4THitsCollection}.
At the end of each run, they are written to a text file
using functions from \code{G4AnalysisManager}.
