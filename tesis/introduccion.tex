\section{Introducción}
La radiación es una herramienta con numerosas aplicaciones.

En la industria, se usa para esterilizar instrumental médico y comida,
alterar propiedades químicas de superficies\cite{clough2001high},
y para muchos tipos de mediciones.
En particular, se usa para ensayos no destructivos
como la radiografía de neutrones\cite{berger_neutron_1960}.

En medicina, la radiación se emplea en diagnóstico para adquirir imágenes del cuerpo,
y en terapia para tratar cáncer y otras enfermedades.
Con la incidencia creciente de esta enfermedad 
y el uso de radiación en 50\% de los
pacientes\cite{symposium_assurance_dosimetry_1994},
gran parte de la población va a ser expuesta a radiación.

Con el desarrollo de estas aplicaciones,
se empezó a tomar conciencia de 
los peligros de la exposición a la radiación,
y la importancia de entender sus efectos en tejidos
para establecer prácticas seguras.

Una parte central de estas prácticas es el uso de dosímetros personales.
Los mismos permiten limitar los tiempos e intensidades de exposición
por debajo de valores riesgosos.

Además de proteger al personal de un hospital o planta,
los dosímetros permiten monitorear de forma precisa 
el tratamiento que recibe un paciente.
Su uso facilita detectar errores de aplicación\cite{noel_detection_1995}.
También permiten verificar la planificación de nuevas
técnicas y así mejorar el estándar de cuidado\cite{essers_vivo_1999}.
Por último, la dosimetría \emph{in vivo} 
abre la puerta a terapias más efectivas:
es posible planificar cada sesión de radiación
en respuesta al resultado de la anterior,
corrigiendo por fallas de alineamiento, calibración 
y cambios en el paciente\cite{wu_application_2006}.

Este tipo de medición demanda dosímetros 
que se presten al uso médico.
Los requerimientos pasan tanto por sus especificaciones técnicas
(sensibilidad, dosis máxima)
como por su costo,
biocompatibilidad, tamaño,
demora en la lectura y simplicidad de uso.

Desde hace mucho tiempo se usan dosímetros basados en 
dispositivos semiconductores discretos.
Los mismos se basan en el mismo principio que una cámara de ionización de aire,
pero miles de veces más sensibles por unidad de volúmen\cite{jones_application_1963}.
Así posibilitan mediciones con mayor resolución espacial.

Más reciente es un tipo de dosímetro que usa técnicas 
provenientes de la fabricación de circuitos integrados 
para obtener dispositivos 
miniaturizables\cite{holmes-siedle_radfet:_1986}.
Actualmente consisten en circuitos que miden el 
cambio en las características eléctricas de un transistor.
Este es un tipo especial de transistor denominado RADFET,
particularmente sensible a la radiación 
debido a su óxido de compuerta muy grueso.

Dentro de los dosímetros integrables 
(que se pueden incorporar con otras funciones en un circuito integrado),
hay gran interés en aquellos fabricados usando, sin modificación,
procesos comerciales para circuitos integrados\cite{lipovetzky_field_2013}
\cite{wang_temperature_2005}
\cite{garcia-moreno_floating_2012}
\cite{dulinski_cmos_2004}.
Esto elimina la posibilidad de optimizar y controlar 
los parámetros del proceso 
para los requerimientos específicos de dosimetría.
A cambio de esa restricción, 
permite integrar circuitería adicional
para procesamiento de señales e interfaz con el mundo exterior,
y aprovechar las economías de escala de los procesos estándar.

En este trabajo diseñamos, construímos y caracterizamos
dos dosímetros fabricados en un proceso estándar CMOS de
\SI{0.6}{\micro\meter}.
El primero es un Active Pixel Sensor,
de estructura similar a un pixel del sensor de una cámara digital.
El volúmen sensible es la zona desierta de portadores 
de un diodo polarizado en inversa.
La radiación incidente en esta zona genera pares electrón-hueco.
El campo eléctrico separa electrones de huecos y los transporta hacia
terminales opuestas del diodo,
con una fracción de ellos desapareciendo por recombinación.
Así la radiación incidente produce una corriente que va
descargando un capacitor.
El cambio en su valor de carga es un indicador de la energía total recibida.

Su ventaja sobre dosímetros tradicionales es la capacidad de resetearlo
instantáneamente de manera electrónica,
simplemente recargando el capacitor a su valor de carga original.

El segundo dosímetro es un Floating Gate Transistor,
semejante al transistor MOS en una celda de memoria Flash.
Su compuerta se encuentra completamente aislada eléctricamente (flotando),
para almacenar una carga colocada antes de la irradiación.
La radiación incidente genera portadores en el aislante que rodea la compuerta
y va descargándola en proporción a la energía capturada.
Luego de la irradiación,
se mide la carga remanente a través de su efecto en las curvas del transistor.
Este dispositivo es idóneo para dicha medición
porque la corriente que atraviesa la compuerta 
(una fuente indeseada de descarga) es insignificante,
a diferencia de los transistores BJT y JFET.
Este dosímetro se destaca por la posibilidad de medir radiación sin suministro
de tensión, ya que la compuerta flotante retiene su carga durante tiempos muy
largos.

Ambos dosímetros explotan la respuesta a radiación de dispositivos 
normalmente utilizados para otros fines. 
Luego de una introducción a la teoría de su funcionamiento,
presentamos el proceso y las consideraciones de diseño,
y los resultados de las mediciones de ambos dosímetros 
comparando con los valores calculados y cos trabajos previos.
