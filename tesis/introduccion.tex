\section{Introduction}
Radiation is a natural phenomenon with wide-ranging uses.

It is used industrially to sterilize food and medical equipment,
to alter surface chemical properties\cite{clough2001high},
and for several measurement techniques.
In particular, it is used for non-destructive assays such as
neutron radiography\cite{berger_neutron_1960}.

In the medical field, radiation is used both in diagnostics,
to image the inside of the body,
and in therapy, for the treatment of cancer and other diseases.
Given the increasing rates of cancer,
and the use of radiation in half of patients
\cite{symposium_assurance_dosimetry_1994},
a significant part of the population will be exposed to radiation
at some point in their lives.

As medical and industrial applications became more widespread,
there was a growing need to understand the dangers of radiation exposure.
This led to the study of radiation's effects in living tissue,
in order to establish safe operating practices.

One major way to ensure safety is routine use of personal dosimeters.
These devices quantify a person's exposure to radiation,
allowing a worker to manage his length and intensity of exposure
in order to stay within safe limits.

In addition to protecting factory or hospital personnel,
dosimeters can be used to precisely monitor a patient's radiation treatment.
This enables the detection of over- or under-dosing errors\cite{noel_detection_1995}.
They can also be used to plan new treatment techniques,
in order to improve the standard of care
\cite{essers_vivo_1999}.
Lastly, \emph{in vivo} dosimetry
opens the door to a more effective type of therapy:
it becomes possible to plan each radiation session
in response to the previous session,
correcting for any misalignments,
equipment drift and changes in the patient\cite{wu_application_2006}.

\emph{In vivo} measurement depend on dosimeters fit for medical use.
This implies not only specific technical requirements
(sensitivity, maximum dose)
but also cost, biocompatibility, size, readout speed and ease of use.

Dosimeters are typically based on discrete semiconductor devices.
Their operating principle is like that of an air ionization chamber,
but with a thousand-fold increase in sensitivity per unit volume
\cite{jones_application_1963}.
This makes it feasible to conduct measurements with higher spatial resolution.

A newer kind of dosimeter employs techniques from integrated circuit fabrication
which yield much smaller sensors\cite{holmes-siedle_radfet:_1986}.
For example, there are circuits which measure 
the shift in a transistor's electrical characteristics
due to radiation.
Typically this is a specialized transistor (RADFET)
whose thick gate oxide makes it specially sensitive to radiation.

There is great interest in building a dosimeter using 
unmodified commercial integrated circuit processes\cite{lipovetzky_field_2013}
\cite{wang_temperature_2005}
\cite{garcia-moreno_floating_2012}
\cite{dulinski_cmos_2004}.
This does not allow the process to be optimized for dosimetry requirements.
However, it opens the door to integrating additional analog and/or digital
circuitry into the sensor.
This would allow interface and signal processing circuitry to be cheaply included in a single die.
In addition, it exploits the economies of scale of standard fabrication processes.

This thesis analyzes the design, construction and characterization
of two dosimeters using a standard \SI{0.6}{\micro\meter} CMOS process.
The first dosimeter is an Active Pixel Sensor,
with a structure resembling a pixel from a digital camera sensor.
The sensitive volume is the depletion region of a reverse-biased diode.
Incoming radiation creates electron-hole pairs,
which are separated out to opposite terminals by the electric field.
This results in a current which discharges a capacitor.
By measuring this discharge, it is possible to find the total energy deposited in the sensor.

Unlike an ionization chamber,
the APS dosimeter requires periodic resetting.
However, this only requires the capacitor to be recharged,
which is almost instantaneous and does not require high voltages.

The second dosimeter is a Floating Gate transistor,
similar to the storage element in a Flash memory cell.
It is a MOSFET whose gate is electrically isolated,
so that it can be pre-charged before irradiating.
Incoming radiation creates electron-hole pairs in the insulation around the floating gate,
discharging it in direct proportion to the deposited energy.
After irradiating, it is possible to determine how much charge remains
by measuring the MOSFET's current-voltage characteristics.
MOS devices are ideally suited for this purpose because they have zero gate current,
so they can store charge for long periods of time.
Floating Gate dosimeters stand out due to their ability to measure radiation without
a voltage supply after the initial charging of the floating gate.

Both dosimeters are based on measuring the radiation response of devices which are normally used for other purposes.
After presenting the theory behind their operation,
this thesis covers the fabrication process and resulting design considerations.
It goes on to report the measurement results for both dosimeters,
which are compared with calculations and with prior works.
