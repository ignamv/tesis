\section{Monte Carlo simulations}
\label{montecarlo}
Interactions between radiation and matter are events such as
\begin{itemize}
    \item Scattering: energy and momentum transfer,
        for example between a photon and an electron
    \item Pair creation: conversion of energy to a
        particle/antiparticle pair.
        For example, when a photon strikes a nucleus and generates an
        electron and a positron.
\end{itemize}
These processes are non-deterministic.
Their results can be described by using a probability distribution, such as
\begin{itemize}
    \item decay probability per unit time (activity), or
    \item differential scattering cross section $\deriv{\sigma}{\Omega}$:
        the scattering probability per unit solid angle,
        for a given direction.
\end{itemize}
In some cases, the non-determinism is inescapable.
For example, in space one must consider the danger caused by
very small probability events with very high energy
incoming particles.

In other cases (industrial and medical applications),
statistical variations tend to cancel out due to the law of large numbers.
It is enough to calculate the expected value of the magnitudes of interest.

A typical problem is calculating the dose in a target
(for example, a circuit or person)
for a given environment and radiation source.
Each particle exiting the source can take part in many interactions,
and can generate multiple secondary particles.
Therefore, the space of final states has high dimensionality.
This makes in infeasible to calculate the probability distribution for all final states.
Therefore, it is not possible to calculate the expected value for the dose
starting from the distribution function.

The Monte Carlo method\cite{roe_probability_1992}
consists of generating random samples of the final states,
and calculating statistics based on those samples.
To generate the random samples,
one simulates a particle's evolution,
choosing at random between possible interactions
according to their relative probabilities.

In practice, this sampling is carried out using one of many
software packages.
They typically come bundled with material databases
with tables characterizing each kind of interaction.
This frees up the user to model the problem geometry,
using 3D modeling software such as FreeCAD.

I used the Geant4 toolkit\cite{allison_geant4_2006},
selecting the particles and processes required for $\beta$ and X radiation,
to calculate the dose for different parts of this thesis.
