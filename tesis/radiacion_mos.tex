\section{Radiation effects in electronic devices}
There are two main applications for the study of radiation effects in electronic devices:
\begin{itemize}
    \item circuit design for use in high-radiation areas (satellites, nuclear reactors), and
    \item radiation measurements.
\end{itemize}
Many types of dosimeters work on the principle of storing charge
produced by ionizing radiation.
This charge can cause voltage and current transients in integrated circuits,
and cumulative damage in the component transistors.

\subsection{Radiation effects in p-n junctions}
\subsubsection{Cumulative damage}
Radiation can displace atoms from the semiconductor lattice,
which create active defects.
These are charged sited which break the lattice's translation symmetry.
\cite{iniewski_radiation_2011}.
When active defects are present in a p-n junction's depletion region,
they act as generation/recombination centers.
In the reverse-bias condition, they increate the production rate of electron-hole pairs,
which increases the reverse leakage current
\cita{bogaert_total_2000}.

This type of damage is mainly caused by heavy ion irradiation,
and is less significant with $\beta$ and $\gamma$ radiation
\cite{knoll_radiation_2010}%p. 397
\cite{liu_electron_1971}.
\subsubsection{Charge transients}
\label{latchup}
When radiation crosses a junction's depletion region,
it releases some of its energy in the production of electron-hole pairs.
The depletion region's electric field
moves the varriers to opposite terminals,
causing a current transient which flows from n to p.
This current adds to the leakage current caused by thermal generation
(spontaneous carrier production),
and the minority carrier current.

Current transients can lead to a failure mode called Latch-Up
\cite{gregory_latch-up_1973}.
The transient can briefly forward-bias a junction.
If there are reverse-biased junctions nearby,
they can collect carriers from the first junction,
forming a parasitic bipolar transistor.
Under the right conditions,
these parasitic structures (\figref{fig:latchup})
can reach a steady state in which
the supply rails are short-circuited.
This can go on until supply voltage is cut off
(for example, by an overcurrent protection circuit),
or until the circuit is destroyed.
\fig{latchup}{figuras/mos/latchup.pdf}{
Parasitic transistors in a standard CMOS process.
The normal stable condition is for both transistors to be cut-off.
The anomalous stable condition is for both transistors to be on,
each supplying base current to the other.
This can destroy the circuit due to excessive current.}
\subsection{Radiation effects in MOS}
Ionization in MOS gate oxides creates carriers which
have multiple damaging effects\cita{oldham_total_2003}. 
\subsubsection{Charge trapping}
Radiation that impacts a device produces electron-hole pairs (\figref{fig:capturamos}).
Some regions of the die are particularly sensitive to the resulting charge:
as was explained previously,
charge in the semiconductor causes cumulative damage and latchup.
When charge is produced in the gate oxide of a gate under positive bias,
electrons drift from the oxide to the gate thanks to their high mobility.
This leaves behind the holes.
Their positive charge in the oxide causes a negative threshold voltage shift
(section~\ref{corrimientovt}).
Holes diffuse slowly towards the Si-SiO$_2$ interface.
A fraction of them cross the interface,
exiting the oxide and partially undoing the $V_T$ shift.
The rest are captured by traps,
where they can remain from hours to years.
As they are released,
the $V_T$ slowly recovers to its original value.
\fig{capturamos}{figuras/radmos/radiacion_mos.pdf}{Charge
trapping process in MOS oxides due to radiation}
\subsubsection{Interface trap creation}
Another consequence of radiation is the creation of interface traps.
These are localized states in the Si-SiO$_2$ interface
with energies in the silicon forbidden band.
They can exchange charge with the silicon,
by capturing or releasing holes or electrons.
The traps' surface charge density changes with the surface Fermi level,
leading to a $V_G$-dependent $V_T$ shift.
\subsubsection{$V_T$ shift}
\label{corrimientovt}
Charges in the gate oxide alter the relation between gate voltage and
surface potential.
This can be analyzed in 1D,
placing the gate at $x=0$ and the semiconductor surface at $x=t_{ox}$:
\begin{align}
    V_g-\psi_s&=\int_0^{t_{ox}}E(x)dx=
    \int_0^{t_{ox}}\left[\frac d{dx}(xE)-x\frac{dE}{dx}\right]dx\\
    &=t_{ox}\mathscr{E}_s-\frac 1{\epsilon_{ox}}\int_0^{t_{ox}}x\rho(x)dx\\
    \mathscr{E}_s &= \frac{V_g-\psi_s+
        \frac 1{\epsilon_{ox}}\int_0^{t_{ox}}x\rho(x)dx}{t_{ox}}.
        \label{eq:corrimiento_vt_mos}
\end{align}
It can be seen that oxide charges cause a shift in the $V_g$
dependence of the device curves.
The closer the charge is to the semiconductor,
the larger the shift.

Interface traps have a $E_F$-dependent surface charge density,
given by
\begin{align*}
    \sigma_{it} &= -e\int_{E_0}^{E_F} D_{it}dE
\end{align*}
with $D_{it}$ the interface trap density per unit energy,
and $E_0$ the value of $E_F$ where charge from donor traps cancels
with charge from acceptor traps.
Interface trap charge alters the device curves:
as $V_G$ and therefore $\psi_s$ vary,
the Fermi level moves through the trap energies,
changing $\sigma_{it}$ and thus shifting $V_T$.
This $V_g$ dependent $\Delta V_T$
manifests as a stretching of all device curves (I-V and C-V)
in the $V_g$ axis.
