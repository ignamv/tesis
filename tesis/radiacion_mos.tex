\section{Efectos de radiación en dispositivos}
El estudio de los efectos de la radiación en dispositivos electrónicos 
tiene dos grandes áreas de aplicación:
\begin{itemize}
    \item el diseño de circuitos para uso en zonas de radiación intensa
        (satélites, reactores nucleares), y
    \item la medición de la radiación a través de su efecto en un dispositivo.
\end{itemize}
Muchos tipos de dosímetros se basan en acumular 
carga por ionización proveniente de radiación.
Esta carga produce transitorios de corriente y tensión en los circuitos 
integrados, y daño acumulativo en los transistores que los componen.

\subsection{Radiación en junturas p-n}
\subsubsection{Daño acumulativo}
La radiación desplaza átomos de la red del semiconductor, 
creando defectos activos.
Estos son sitios cargados donde se rompe la periodicidad de la red
\cite{iniewski_radiation_2011}.
Los defectos localizados en la zona desierta de una juntura p-n
actúan como centros de generación/recombinación.
Su presencia facilita la creación de pares electrón-hueco, 
aumentando la corriente de fuga en inversa
\cita{bogaert_total_2000}.

Este tipo de daño se da principalmente al irradiar con iones pesados,
y tiene menor importancia con partículas $\beta$ o $\gamma$
\cite{knoll_radiation_2010}%p. 397
\cite{liu_electron_1971}.
\subsubsection{Transitorios de carga}
\label{latchup}
Cuando la radiación alcanza la zona desierta de una juntura,
deposita parte de su energía creando pares electrón-hueco.
El campo eléctrico lleva los portadores creados a terminales opuestas,
produciendo un transitorio de corriente que fluye del lado n al p.
Éste se suma a la corriente de pérdida que tienen las junturas
debido a generación térmica (creación espontánea de pares),
y a la corriente de portadores minoritarios.

Los transitorios pueden llevar a un modo de falla llamado Latch-Up
\cite{gregory_latch-up_1973}.
El transitorio polariza una juntura en directa brevemente.
Si hay otras junturas cercanas en inversa,
pueden colectar los portadores de la primera juntura 
formando un transistor bipolar parásito.
Si se dan las condiciones adecuadas,
un conjunto de estas estructuras parásitas (\figref{fig:latchup})
llega a un estado estable
en el que cortocircuitan los rieles de alimentación.
Esto continúa hasta que se apague la alimentación 
(por ejemplo si hay protección contra exceso de corriente)
o se destruya el circuito.
\fig{latchup}{figuras/mos/latchup.pdf}{
Transistores parásitos presentes en un proceso CMOS estándar.
La condición estable normal es ambos transistores apagados.
La condición estable anómala es ambos transistores prendidos,
cada uno suministrando la corriente de base del otro.
Esta última condición puede destruir al circuito por exceso de corriente.}
\subsection{Radiación en MOS}
La ionización del óxido de las estructuras MOS crea portadores 
con distintos efectos deletéreos\cita{oldham_total_2003}. 
\subsubsection{Captura de carga}
La radiación que incide en el dispositivo 
produce una cantidad de pares electrón-hueco (\figref{fig:capturamos}).
Algunas regiones son particularmente sensibles a esta carga generada.
Cuando la ionización sucede en el óxido de un gate polarizado positivamente,
los electrones derivan del óxido al gate gracias a su alta movilidad.
Así quedan sólo los huecos.
La carga positiva en el óxido reduce la tensión umbral 
(ver sección~\ref{corrimientovt}).
Los huecos se difunden lentamente hacia la interfaz Si-SiO$_2$.
Una fracción la atraviesa, 
saliendo del óxido y restaurando parcialmente el $V_T$.
El resto es capturado por trampas,
que retienen carga por tiempos largos (de horas a años).
A medida que esta carga se libera, 
lleva a una recuperación lenta del $V_T$.
\fig{capturamos}{figuras/radmos/radiacion_mos.pdf}{Mecanismo de captura de
carga en óxidos de MOS debido a radiación.}
\subsubsection{Creación de trampas de interfaz}
El otro efecto de la radiación es crear trampas de interfaz.
Estos son estados localizados en la interfaz Si-SiO$_2$ con
energías entre la banda de valencia y de conducción del Si.
Pueden intercambiar carga con el Si,
capturando o liberando tanto electrones como huecos.
La densidad superficial de carga debido a estas trampas varía con el nivel de
Fermi en la superficie, produciendo un $\Delta V_T$ dependiente de $V_G$.
\subsubsection{Corrimiento de $V_T$}
\label{corrimientovt}
La carga en el óxido de gate altera la relación 
entre la tensión aplicada al gate y el campo en la interfaz Si-SiO$_2$.
Analizamos este fenómeno en 1D con $x=0$ en el gate y $x=t_{ox}$
en el semiconductor:
\begin{align*}
    V_g-\psi_s&=\int_0^{t_{ox}}E(x)dx=
    \int_0^{t_{ox}}\left[\frac d{dx}(xE)-x\frac{dE}{dx}\right]dx\\
    &=t_{ox}\mathscr{E}_s-\frac 1{\epsilon_{ox}}\int_0^{t_{ox}}x\rho(x)dx\\
    \mathscr{E}_s &= \frac{V_g-\psi_s+
        \frac 1{\epsilon_{ox}}\int_0^{t_{ox}}x\rho(x)dx}{t_{ox}}.
\end{align*}
Se ve que la carga desplaza las curvas del dispositivo 
en su dependencia con $V_g$.
Mientras más cerca esté la carga al semiconductor,
mayor es este corrimiento.

Las trampas de interfaz contienen una densidad superficial de carga dependiente
de $E_F$, dada por
\begin{align*}
    \sigma_{it} &= -e\int_{E_0}^{E_F} D_{it}dE
\end{align*}
siendo $D_{it}$ la densidad de trampas por unidad de energía,
y $E_0$ el valor de $E_F$ para el cual se cancela la carga de las trampas
donantes con la carga de las trampas aceptoras.
Esta carga produce una deformación de las curvas del dispositivo,
porque para cada $\psi_s$ se tiene un $\sigma_{it}$ distinto que produce un
corrimiento distinto.
Se observa como un estiramiento de las curvas.
